\documentclass[]{article}
\usepackage{lmodern}
\usepackage{amssymb,amsmath}
\usepackage{ifxetex,ifluatex}
\usepackage{fixltx2e} % provides \textsubscript
\ifnum 0\ifxetex 1\fi\ifluatex 1\fi=0 % if pdftex
  \usepackage[T1]{fontenc}
  \usepackage[utf8]{inputenc}
\else % if luatex or xelatex
  \ifxetex
    \usepackage{mathspec}
  \else
    \usepackage{fontspec}
  \fi
  \defaultfontfeatures{Ligatures=TeX,Scale=MatchLowercase}
\fi
% use upquote if available, for straight quotes in verbatim environments
\IfFileExists{upquote.sty}{\usepackage{upquote}}{}
% use microtype if available
\IfFileExists{microtype.sty}{%
\usepackage{microtype}
\UseMicrotypeSet[protrusion]{basicmath} % disable protrusion for tt fonts
}{}
\usepackage[margin=2.54cm]{geometry}
\usepackage{hyperref}
\hypersetup{unicode=true,
            pdftitle={Assignment 8: Time Series Analysis},
            pdfauthor={Claire Mullaney},
            pdfborder={0 0 0},
            breaklinks=true}
\urlstyle{same}  % don't use monospace font for urls
\usepackage{color}
\usepackage{fancyvrb}
\newcommand{\VerbBar}{|}
\newcommand{\VERB}{\Verb[commandchars=\\\{\}]}
\DefineVerbatimEnvironment{Highlighting}{Verbatim}{commandchars=\\\{\}}
% Add ',fontsize=\small' for more characters per line
\usepackage{framed}
\definecolor{shadecolor}{RGB}{248,248,248}
\newenvironment{Shaded}{\begin{snugshade}}{\end{snugshade}}
\newcommand{\AlertTok}[1]{\textcolor[rgb]{0.94,0.16,0.16}{#1}}
\newcommand{\AnnotationTok}[1]{\textcolor[rgb]{0.56,0.35,0.01}{\textbf{\textit{#1}}}}
\newcommand{\AttributeTok}[1]{\textcolor[rgb]{0.77,0.63,0.00}{#1}}
\newcommand{\BaseNTok}[1]{\textcolor[rgb]{0.00,0.00,0.81}{#1}}
\newcommand{\BuiltInTok}[1]{#1}
\newcommand{\CharTok}[1]{\textcolor[rgb]{0.31,0.60,0.02}{#1}}
\newcommand{\CommentTok}[1]{\textcolor[rgb]{0.56,0.35,0.01}{\textit{#1}}}
\newcommand{\CommentVarTok}[1]{\textcolor[rgb]{0.56,0.35,0.01}{\textbf{\textit{#1}}}}
\newcommand{\ConstantTok}[1]{\textcolor[rgb]{0.00,0.00,0.00}{#1}}
\newcommand{\ControlFlowTok}[1]{\textcolor[rgb]{0.13,0.29,0.53}{\textbf{#1}}}
\newcommand{\DataTypeTok}[1]{\textcolor[rgb]{0.13,0.29,0.53}{#1}}
\newcommand{\DecValTok}[1]{\textcolor[rgb]{0.00,0.00,0.81}{#1}}
\newcommand{\DocumentationTok}[1]{\textcolor[rgb]{0.56,0.35,0.01}{\textbf{\textit{#1}}}}
\newcommand{\ErrorTok}[1]{\textcolor[rgb]{0.64,0.00,0.00}{\textbf{#1}}}
\newcommand{\ExtensionTok}[1]{#1}
\newcommand{\FloatTok}[1]{\textcolor[rgb]{0.00,0.00,0.81}{#1}}
\newcommand{\FunctionTok}[1]{\textcolor[rgb]{0.00,0.00,0.00}{#1}}
\newcommand{\ImportTok}[1]{#1}
\newcommand{\InformationTok}[1]{\textcolor[rgb]{0.56,0.35,0.01}{\textbf{\textit{#1}}}}
\newcommand{\KeywordTok}[1]{\textcolor[rgb]{0.13,0.29,0.53}{\textbf{#1}}}
\newcommand{\NormalTok}[1]{#1}
\newcommand{\OperatorTok}[1]{\textcolor[rgb]{0.81,0.36,0.00}{\textbf{#1}}}
\newcommand{\OtherTok}[1]{\textcolor[rgb]{0.56,0.35,0.01}{#1}}
\newcommand{\PreprocessorTok}[1]{\textcolor[rgb]{0.56,0.35,0.01}{\textit{#1}}}
\newcommand{\RegionMarkerTok}[1]{#1}
\newcommand{\SpecialCharTok}[1]{\textcolor[rgb]{0.00,0.00,0.00}{#1}}
\newcommand{\SpecialStringTok}[1]{\textcolor[rgb]{0.31,0.60,0.02}{#1}}
\newcommand{\StringTok}[1]{\textcolor[rgb]{0.31,0.60,0.02}{#1}}
\newcommand{\VariableTok}[1]{\textcolor[rgb]{0.00,0.00,0.00}{#1}}
\newcommand{\VerbatimStringTok}[1]{\textcolor[rgb]{0.31,0.60,0.02}{#1}}
\newcommand{\WarningTok}[1]{\textcolor[rgb]{0.56,0.35,0.01}{\textbf{\textit{#1}}}}
\usepackage{graphicx,grffile}
\makeatletter
\def\maxwidth{\ifdim\Gin@nat@width>\linewidth\linewidth\else\Gin@nat@width\fi}
\def\maxheight{\ifdim\Gin@nat@height>\textheight\textheight\else\Gin@nat@height\fi}
\makeatother
% Scale images if necessary, so that they will not overflow the page
% margins by default, and it is still possible to overwrite the defaults
% using explicit options in \includegraphics[width, height, ...]{}
\setkeys{Gin}{width=\maxwidth,height=\maxheight,keepaspectratio}
\IfFileExists{parskip.sty}{%
\usepackage{parskip}
}{% else
\setlength{\parindent}{0pt}
\setlength{\parskip}{6pt plus 2pt minus 1pt}
}
\setlength{\emergencystretch}{3em}  % prevent overfull lines
\providecommand{\tightlist}{%
  \setlength{\itemsep}{0pt}\setlength{\parskip}{0pt}}
\setcounter{secnumdepth}{0}
% Redefines (sub)paragraphs to behave more like sections
\ifx\paragraph\undefined\else
\let\oldparagraph\paragraph
\renewcommand{\paragraph}[1]{\oldparagraph{#1}\mbox{}}
\fi
\ifx\subparagraph\undefined\else
\let\oldsubparagraph\subparagraph
\renewcommand{\subparagraph}[1]{\oldsubparagraph{#1}\mbox{}}
\fi

%%% Use protect on footnotes to avoid problems with footnotes in titles
\let\rmarkdownfootnote\footnote%
\def\footnote{\protect\rmarkdownfootnote}

%%% Change title format to be more compact
\usepackage{titling}

% Create subtitle command for use in maketitle
\providecommand{\subtitle}[1]{
  \posttitle{
    \begin{center}\large#1\end{center}
    }
}

\setlength{\droptitle}{-2em}

  \title{Assignment 8: Time Series Analysis}
    \pretitle{\vspace{\droptitle}\centering\huge}
  \posttitle{\par}
    \author{Claire Mullaney}
    \preauthor{\centering\large\emph}
  \postauthor{\par}
    \date{}
    \predate{}\postdate{}
  

\begin{document}
\maketitle

\hypertarget{overview}{%
\subsection{OVERVIEW}\label{overview}}

This exercise accompanies the lessons in Environmental Data Analytics on
time series analysis.

\hypertarget{directions}{%
\subsection{Directions}\label{directions}}

\begin{enumerate}
\def\labelenumi{\arabic{enumi}.}
\tightlist
\item
  Change ``Student Name'' on line 3 (above) with your name.
\item
  Work through the steps, \textbf{creating code and output} that fulfill
  each instruction.
\item
  Be sure to \textbf{answer the questions} in this assignment document.
\item
  When you have completed the assignment, \textbf{Knit} the text and
  code into a single PDF file.
\item
  After Knitting, submit the completed exercise (PDF file) to the
  dropbox in Sakai. Add your last name into the file name (e.g.,
  ``Salk\_A06\_GLMs\_Week1.Rmd'') prior to submission.
\end{enumerate}

The completed exercise is due on Tuesday, March 3 at 1:00 pm.

\hypertarget{set-up}{%
\subsection{Set up}\label{set-up}}

\begin{enumerate}
\def\labelenumi{\arabic{enumi}.}
\tightlist
\item
  Set up your session:
\end{enumerate}

\begin{itemize}
\tightlist
\item
  Check your working directory
\item
  Load the tidyverse, lubridate, zoo, and trend packages
\item
  Set your ggplot theme
\item
  Import the ten datasets from the Ozone\_TimeSeries folder in the Raw
  data folder. These contain ozone concentrations at Garinger High
  School in North Carolina from 2010-2019 (the EPA air database only
  allows downloads for one year at a time). Call these
  GaringerOzone201*, with the star filled in with the appropriate year
  in each of ten cases.
\end{itemize}

\begin{Shaded}
\begin{Highlighting}[]
\CommentTok{#Checking working directory and loading packages}

\KeywordTok{getwd}\NormalTok{()}
\end{Highlighting}
\end{Shaded}

\begin{verbatim}
## [1] "/Users/clairemullaney/Desktop/ENV 872/Environmental_Data_Analytics_2020"
\end{verbatim}

\begin{Shaded}
\begin{Highlighting}[]
\KeywordTok{library}\NormalTok{(tidyverse)}
\KeywordTok{library}\NormalTok{(lubridate)}
\KeywordTok{library}\NormalTok{(zoo)}
\KeywordTok{library}\NormalTok{(trend)}

\CommentTok{#Setting default ggplot theme}
\NormalTok{deftheme <-}\StringTok{ }\KeywordTok{theme_classic}\NormalTok{(}\DataTypeTok{base_size =} \DecValTok{14}\NormalTok{) }\OperatorTok{+}
\StringTok{  }\KeywordTok{theme}\NormalTok{(}\DataTypeTok{axis.text =} \KeywordTok{element_text}\NormalTok{(}\DataTypeTok{color =} \StringTok{"black"}\NormalTok{), }
        \DataTypeTok{legend.position =} \StringTok{"right"}\NormalTok{)}
\KeywordTok{theme_set}\NormalTok{(deftheme)}

\CommentTok{#Importing datasets and placing them into a list}
\NormalTok{Ozone_files <-}\StringTok{ }\KeywordTok{list.files}\NormalTok{(}\DataTypeTok{path =} \StringTok{"./Data/Raw/Ozone_TimeSeries/"}\NormalTok{, }
                         \DataTypeTok{pattern=}\StringTok{"*.csv"}\NormalTok{, }\DataTypeTok{full.names=}\OtherTok{TRUE}\NormalTok{)}

\NormalTok{Ozone_files_named <-}\StringTok{ }\KeywordTok{list}\NormalTok{()}

\ControlFlowTok{for}\NormalTok{(i }\ControlFlowTok{in} \DecValTok{1}\OperatorTok{:}\KeywordTok{length}\NormalTok{(Ozone_files)) \{}
\NormalTok{  Ozone_files_named[[i]] <-}\StringTok{ }\KeywordTok{assign}\NormalTok{(}\KeywordTok{paste}\NormalTok{(}\StringTok{"GaringerOzone201"}\NormalTok{, i}\DecValTok{-1}\NormalTok{ , }\DataTypeTok{sep =} \StringTok{""}\NormalTok{), }\KeywordTok{read.csv}\NormalTok{(Ozone_files[i]))}
\NormalTok{\}}
\end{Highlighting}
\end{Shaded}

\hypertarget{wrangle}{%
\subsection{Wrangle}\label{wrangle}}

\begin{enumerate}
\def\labelenumi{\arabic{enumi}.}
\setcounter{enumi}{1}
\item
  Combine your ten datasets into one dataset called GaringerOzone. Think
  about whether you should use a join or a row bind.
\item
  Set your date column as a date class.
\item
  Wrangle your dataset so that it only contains the columns Date,
  Daily.Max.8.hour.Ozone.Concentration, and DAILY\_AQI\_VALUE.
\item
  Notice there are a few days in each year that are missing ozone
  concentrations. We want to generate a daily dataset, so we will need
  to fill in any missing days with NA. Create a new data frame that
  contains a sequence of dates from 2010-01-01 to 2019-12-31 (hint:
  \texttt{as.data.frame(seq())}). Call this new data frame Days. Rename
  the column name in Days to ``Date''.
\item
  Use a \texttt{left\_join} to comine the data frames. Specify the
  correct order of data frames within this function so that the final
  dimensions are 3652 rows and 3 columns. Call your combined data frame
  GaringerOzone.
\end{enumerate}

\begin{Shaded}
\begin{Highlighting}[]
\CommentTok{# 2 }
\CommentTok{#Checking column names to decide whether to use join vs. rbind}
\KeywordTok{lapply}\NormalTok{(Ozone_files_named, colnames)}
\end{Highlighting}
\end{Shaded}

\begin{verbatim}
## [[1]]
##  [1] "Date"                                
##  [2] "Source"                              
##  [3] "Site.ID"                             
##  [4] "POC"                                 
##  [5] "Daily.Max.8.hour.Ozone.Concentration"
##  [6] "UNITS"                               
##  [7] "DAILY_AQI_VALUE"                     
##  [8] "Site.Name"                           
##  [9] "DAILY_OBS_COUNT"                     
## [10] "PERCENT_COMPLETE"                    
## [11] "AQS_PARAMETER_CODE"                  
## [12] "AQS_PARAMETER_DESC"                  
## [13] "CBSA_CODE"                           
## [14] "CBSA_NAME"                           
## [15] "STATE_CODE"                          
## [16] "STATE"                               
## [17] "COUNTY_CODE"                         
## [18] "COUNTY"                              
## [19] "SITE_LATITUDE"                       
## [20] "SITE_LONGITUDE"                      
## 
## [[2]]
##  [1] "Date"                                
##  [2] "Source"                              
##  [3] "Site.ID"                             
##  [4] "POC"                                 
##  [5] "Daily.Max.8.hour.Ozone.Concentration"
##  [6] "UNITS"                               
##  [7] "DAILY_AQI_VALUE"                     
##  [8] "Site.Name"                           
##  [9] "DAILY_OBS_COUNT"                     
## [10] "PERCENT_COMPLETE"                    
## [11] "AQS_PARAMETER_CODE"                  
## [12] "AQS_PARAMETER_DESC"                  
## [13] "CBSA_CODE"                           
## [14] "CBSA_NAME"                           
## [15] "STATE_CODE"                          
## [16] "STATE"                               
## [17] "COUNTY_CODE"                         
## [18] "COUNTY"                              
## [19] "SITE_LATITUDE"                       
## [20] "SITE_LONGITUDE"                      
## 
## [[3]]
##  [1] "Date"                                
##  [2] "Source"                              
##  [3] "Site.ID"                             
##  [4] "POC"                                 
##  [5] "Daily.Max.8.hour.Ozone.Concentration"
##  [6] "UNITS"                               
##  [7] "DAILY_AQI_VALUE"                     
##  [8] "Site.Name"                           
##  [9] "DAILY_OBS_COUNT"                     
## [10] "PERCENT_COMPLETE"                    
## [11] "AQS_PARAMETER_CODE"                  
## [12] "AQS_PARAMETER_DESC"                  
## [13] "CBSA_CODE"                           
## [14] "CBSA_NAME"                           
## [15] "STATE_CODE"                          
## [16] "STATE"                               
## [17] "COUNTY_CODE"                         
## [18] "COUNTY"                              
## [19] "SITE_LATITUDE"                       
## [20] "SITE_LONGITUDE"                      
## 
## [[4]]
##  [1] "Date"                                
##  [2] "Source"                              
##  [3] "Site.ID"                             
##  [4] "POC"                                 
##  [5] "Daily.Max.8.hour.Ozone.Concentration"
##  [6] "UNITS"                               
##  [7] "DAILY_AQI_VALUE"                     
##  [8] "Site.Name"                           
##  [9] "DAILY_OBS_COUNT"                     
## [10] "PERCENT_COMPLETE"                    
## [11] "AQS_PARAMETER_CODE"                  
## [12] "AQS_PARAMETER_DESC"                  
## [13] "CBSA_CODE"                           
## [14] "CBSA_NAME"                           
## [15] "STATE_CODE"                          
## [16] "STATE"                               
## [17] "COUNTY_CODE"                         
## [18] "COUNTY"                              
## [19] "SITE_LATITUDE"                       
## [20] "SITE_LONGITUDE"                      
## 
## [[5]]
##  [1] "Date"                                
##  [2] "Source"                              
##  [3] "Site.ID"                             
##  [4] "POC"                                 
##  [5] "Daily.Max.8.hour.Ozone.Concentration"
##  [6] "UNITS"                               
##  [7] "DAILY_AQI_VALUE"                     
##  [8] "Site.Name"                           
##  [9] "DAILY_OBS_COUNT"                     
## [10] "PERCENT_COMPLETE"                    
## [11] "AQS_PARAMETER_CODE"                  
## [12] "AQS_PARAMETER_DESC"                  
## [13] "CBSA_CODE"                           
## [14] "CBSA_NAME"                           
## [15] "STATE_CODE"                          
## [16] "STATE"                               
## [17] "COUNTY_CODE"                         
## [18] "COUNTY"                              
## [19] "SITE_LATITUDE"                       
## [20] "SITE_LONGITUDE"                      
## 
## [[6]]
##  [1] "Date"                                
##  [2] "Source"                              
##  [3] "Site.ID"                             
##  [4] "POC"                                 
##  [5] "Daily.Max.8.hour.Ozone.Concentration"
##  [6] "UNITS"                               
##  [7] "DAILY_AQI_VALUE"                     
##  [8] "Site.Name"                           
##  [9] "DAILY_OBS_COUNT"                     
## [10] "PERCENT_COMPLETE"                    
## [11] "AQS_PARAMETER_CODE"                  
## [12] "AQS_PARAMETER_DESC"                  
## [13] "CBSA_CODE"                           
## [14] "CBSA_NAME"                           
## [15] "STATE_CODE"                          
## [16] "STATE"                               
## [17] "COUNTY_CODE"                         
## [18] "COUNTY"                              
## [19] "SITE_LATITUDE"                       
## [20] "SITE_LONGITUDE"                      
## 
## [[7]]
##  [1] "Date"                                
##  [2] "Source"                              
##  [3] "Site.ID"                             
##  [4] "POC"                                 
##  [5] "Daily.Max.8.hour.Ozone.Concentration"
##  [6] "UNITS"                               
##  [7] "DAILY_AQI_VALUE"                     
##  [8] "Site.Name"                           
##  [9] "DAILY_OBS_COUNT"                     
## [10] "PERCENT_COMPLETE"                    
## [11] "AQS_PARAMETER_CODE"                  
## [12] "AQS_PARAMETER_DESC"                  
## [13] "CBSA_CODE"                           
## [14] "CBSA_NAME"                           
## [15] "STATE_CODE"                          
## [16] "STATE"                               
## [17] "COUNTY_CODE"                         
## [18] "COUNTY"                              
## [19] "SITE_LATITUDE"                       
## [20] "SITE_LONGITUDE"                      
## 
## [[8]]
##  [1] "Date"                                
##  [2] "Source"                              
##  [3] "Site.ID"                             
##  [4] "POC"                                 
##  [5] "Daily.Max.8.hour.Ozone.Concentration"
##  [6] "UNITS"                               
##  [7] "DAILY_AQI_VALUE"                     
##  [8] "Site.Name"                           
##  [9] "DAILY_OBS_COUNT"                     
## [10] "PERCENT_COMPLETE"                    
## [11] "AQS_PARAMETER_CODE"                  
## [12] "AQS_PARAMETER_DESC"                  
## [13] "CBSA_CODE"                           
## [14] "CBSA_NAME"                           
## [15] "STATE_CODE"                          
## [16] "STATE"                               
## [17] "COUNTY_CODE"                         
## [18] "COUNTY"                              
## [19] "SITE_LATITUDE"                       
## [20] "SITE_LONGITUDE"                      
## 
## [[9]]
##  [1] "Date"                                
##  [2] "Source"                              
##  [3] "Site.ID"                             
##  [4] "POC"                                 
##  [5] "Daily.Max.8.hour.Ozone.Concentration"
##  [6] "UNITS"                               
##  [7] "DAILY_AQI_VALUE"                     
##  [8] "Site.Name"                           
##  [9] "DAILY_OBS_COUNT"                     
## [10] "PERCENT_COMPLETE"                    
## [11] "AQS_PARAMETER_CODE"                  
## [12] "AQS_PARAMETER_DESC"                  
## [13] "CBSA_CODE"                           
## [14] "CBSA_NAME"                           
## [15] "STATE_CODE"                          
## [16] "STATE"                               
## [17] "COUNTY_CODE"                         
## [18] "COUNTY"                              
## [19] "SITE_LATITUDE"                       
## [20] "SITE_LONGITUDE"                      
## 
## [[10]]
##  [1] "Date"                                
##  [2] "Source"                              
##  [3] "Site.ID"                             
##  [4] "POC"                                 
##  [5] "Daily.Max.8.hour.Ozone.Concentration"
##  [6] "UNITS"                               
##  [7] "DAILY_AQI_VALUE"                     
##  [8] "Site.Name"                           
##  [9] "DAILY_OBS_COUNT"                     
## [10] "PERCENT_COMPLETE"                    
## [11] "AQS_PARAMETER_CODE"                  
## [12] "AQS_PARAMETER_DESC"                  
## [13] "CBSA_CODE"                           
## [14] "CBSA_NAME"                           
## [15] "STATE_CODE"                          
## [16] "STATE"                               
## [17] "COUNTY_CODE"                         
## [18] "COUNTY"                              
## [19] "SITE_LATITUDE"                       
## [20] "SITE_LONGITUDE"
\end{verbatim}

\begin{Shaded}
\begin{Highlighting}[]
\CommentTok{#Combining datasets into one using rbind_list; rbind_list was used so the dataframes could be bound from the list created above. To use just rbind, the code would have looked like this:}

\CommentTok{#GaringerOzone <- rbind(GaringerOzone2010, GaringerOzone2011, }
                            \CommentTok{#GaringerOzone2012, GaringerOzone2013,}
                            \CommentTok{#GaringerOzone2014, GaringerOzone2015,}
                            \CommentTok{#GaringerOzone2016, GaringerOzone2017,}
                            \CommentTok{#GaringerOzone2018,GaringerOzone2019)}

\NormalTok{GaringerOzone <-}\StringTok{ }\KeywordTok{rbind_list}\NormalTok{(Ozone_files_named)}
\end{Highlighting}
\end{Shaded}

\begin{verbatim}
## Warning: 'rbind_list' is deprecated.
## Use 'bind_rows()' instead.
## See help("Deprecated")
\end{verbatim}

\begin{verbatim}
## Warning in bind_rows_(list_or_dots(...), id = NULL): Unequal factor levels:
## coercing to character
\end{verbatim}

\begin{verbatim}
## Warning in bind_rows_(list_or_dots(...), id = NULL): binding character and
## factor vector, coercing into character vector

## Warning in bind_rows_(list_or_dots(...), id = NULL): binding character and
## factor vector, coercing into character vector

## Warning in bind_rows_(list_or_dots(...), id = NULL): binding character and
## factor vector, coercing into character vector

## Warning in bind_rows_(list_or_dots(...), id = NULL): binding character and
## factor vector, coercing into character vector

## Warning in bind_rows_(list_or_dots(...), id = NULL): binding character and
## factor vector, coercing into character vector

## Warning in bind_rows_(list_or_dots(...), id = NULL): binding character and
## factor vector, coercing into character vector

## Warning in bind_rows_(list_or_dots(...), id = NULL): binding character and
## factor vector, coercing into character vector

## Warning in bind_rows_(list_or_dots(...), id = NULL): binding character and
## factor vector, coercing into character vector

## Warning in bind_rows_(list_or_dots(...), id = NULL): binding character and
## factor vector, coercing into character vector

## Warning in bind_rows_(list_or_dots(...), id = NULL): binding character and
## factor vector, coercing into character vector
\end{verbatim}

\begin{verbatim}
## Warning in bind_rows_(list_or_dots(...), id = NULL): Unequal factor levels:
## coercing to character
\end{verbatim}

\begin{verbatim}
## Warning in bind_rows_(list_or_dots(...), id = NULL): binding character and
## factor vector, coercing into character vector

## Warning in bind_rows_(list_or_dots(...), id = NULL): binding character and
## factor vector, coercing into character vector
\end{verbatim}

\begin{Shaded}
\begin{Highlighting}[]
\CommentTok{# 3}
\CommentTok{#Changing date column to date format}
\NormalTok{GaringerOzone}\OperatorTok{$}\NormalTok{Date <-}\StringTok{ }\KeywordTok{as.Date}\NormalTok{(GaringerOzone}\OperatorTok{$}\NormalTok{Date, }\DataTypeTok{format =} \StringTok{"%m/%d/%Y"}\NormalTok{)}

\CommentTok{# 4}
\CommentTok{#Selecting necessary columns}
\NormalTok{GaringerOzone.wr <-}\StringTok{ }\NormalTok{GaringerOzone }\OperatorTok
\StringTok{  }\KeywordTok{select}\NormalTok{(Date, Daily.Max.}\FloatTok{8.}\NormalTok{hour.Ozone.Concentration, DAILY_AQI_VALUE)}

\CommentTok{# 5}
\CommentTok{#Creating a days data frame with a column called "Date"}
\NormalTok{Days <-}\StringTok{ }\KeywordTok{as.data.frame}\NormalTok{(}\KeywordTok{seq}\NormalTok{(}\DataTypeTok{from =} \KeywordTok{as.Date}\NormalTok{(}\StringTok{"2010-01-01"}\NormalTok{), }\DataTypeTok{to =} 
                            \KeywordTok{as.Date}\NormalTok{(}\StringTok{"2019-12-31"}\NormalTok{), }\DataTypeTok{by =} \DecValTok{1}\NormalTok{))}

\NormalTok{Days <-}\StringTok{ }\KeywordTok{rename}\NormalTok{(Days, }\StringTok{"Date"}\NormalTok{ =}\StringTok{ "seq(from = as.Date(}\CharTok{\textbackslash{}"}\StringTok{2010-01-01}\CharTok{\textbackslash{}"}\StringTok{), to = as.Date(}\CharTok{\textbackslash{}"}\StringTok{2019-12-31}\CharTok{\textbackslash{}"}\StringTok{), by = 1)"}\NormalTok{)}
 
\CommentTok{# 6}
\CommentTok{#Joining the data frames and checking the dimensions}
\NormalTok{GaringerOzone <-}\StringTok{ }\KeywordTok{left_join}\NormalTok{(Days, GaringerOzone.wr)}
\end{Highlighting}
\end{Shaded}

\begin{verbatim}
## Joining, by = "Date"
\end{verbatim}

\begin{Shaded}
\begin{Highlighting}[]
\KeywordTok{dim}\NormalTok{(GaringerOzone)}
\end{Highlighting}
\end{Shaded}

\begin{verbatim}
## [1] 3652    3
\end{verbatim}

\hypertarget{visualize}{%
\subsection{Visualize}\label{visualize}}

\begin{enumerate}
\def\labelenumi{\arabic{enumi}.}
\setcounter{enumi}{6}
\tightlist
\item
  Create a ggplot depicting ozone concentrations over time. In this
  case, we will plot actual concentrations in ppm, not AQI values.
  Format your axes accordingly.
\end{enumerate}

\begin{Shaded}
\begin{Highlighting}[]
\CommentTok{#Plotting ozone concentrations over time}
\KeywordTok{ggplot}\NormalTok{(}\DataTypeTok{data =}\NormalTok{ GaringerOzone, }
       \KeywordTok{aes}\NormalTok{(}\DataTypeTok{x =}\NormalTok{ Date, }\DataTypeTok{y =}\NormalTok{ Daily.Max.}\FloatTok{8.}\NormalTok{hour.Ozone.Concentration)) }\OperatorTok{+}
\StringTok{  }\KeywordTok{geom_line}\NormalTok{() }\OperatorTok{+}
\StringTok{  }\KeywordTok{labs}\NormalTok{(}\DataTypeTok{y =} \StringTok{"Max 8 hour Ozone Conc (ppm)"}\NormalTok{)}
\end{Highlighting}
\end{Shaded}

\includegraphics{A08_TimeSeries_files/figure-latex/unnamed-chunk-3-1.pdf}

\hypertarget{time-series-analysis}{%
\subsection{Time Series Analysis}\label{time-series-analysis}}

Study question: Have ozone concentrations changed over the 2010s at this
station?

\begin{enumerate}
\def\labelenumi{\arabic{enumi}.}
\setcounter{enumi}{7}
\tightlist
\item
  Use a linear interpolation to fill in missing daily data for ozone
  concentration. Why didn't we use a piecewise constant or spline
  interpolation?
\end{enumerate}

\begin{quote}
Answer: ???
\end{quote}

\begin{enumerate}
\def\labelenumi{\arabic{enumi}.}
\setcounter{enumi}{8}
\item
  Create a new data frame called GaringerOzone.monthly that contains
  aggregated data: mean ozone concentrations for each month. In your
  pipe, you will need to first add columns for year and month to form
  the groupings. In a separate line of code, create a new Date column
  with each month-year combination being set as the first day of the
  month (this is for graphing purposes only)
\item
  Generate a time series called GaringerOzone.monthly.ts, with a monthly
  frequency that specifies the correct start and end dates.
\item
  Run a time series analysis. In this case the seasonal Mann-Kendall is
  most appropriate; why is this?
\end{enumerate}

\begin{quote}
Answer: ???
\end{quote}

\begin{enumerate}
\def\labelenumi{\arabic{enumi}.}
\setcounter{enumi}{11}
\item
  To figure out the slope of the trend, run the function
  \texttt{sea.sens.slope} on the time series dataset.
\item
  Create a plot depicting mean monthly ozone concentrations over time,
  with both a geom\_point and a geom\_line layer. No need to add a line
  for the seasonal Sen's slope; this is difficult to apply to a graph
  with time as the x axis. Edit your axis labels accordingly.
\end{enumerate}

\begin{Shaded}
\begin{Highlighting}[]
\CommentTok{# 8}
\CommentTok{#Linear interpolation}
\NormalTok{GaringerOzone}\OperatorTok{$}\NormalTok{Daily.Max.}\FloatTok{8.}\NormalTok{hour.Ozone.Concentration <-}\StringTok{ }\KeywordTok{na.approx}\NormalTok{(GaringerOzone}\OperatorTok{$}\NormalTok{Daily.Max.}\FloatTok{8.}\NormalTok{hour.Ozone.Concentration)}

\CommentTok{# 9}
\CommentTok{#Adding Month and Year columns, grouping the df by month/year, and summarizing ozone concentrations by month/year}
\NormalTok{GaringerOzone.monthly <-}\StringTok{ }\NormalTok{GaringerOzone }\OperatorTok
\KeywordTok{mutate}\NormalTok{(}\DataTypeTok{Year =} \KeywordTok{year}\NormalTok{(Date), }\DataTypeTok{Month =} \KeywordTok{month}\NormalTok{(Date)) }\OperatorTok
\StringTok{  }\KeywordTok{group_by}\NormalTok{(Year, Month) }\OperatorTok
\StringTok{  }\KeywordTok{summarize}\NormalTok{(}\DataTypeTok{Mean.Ozone.Conc =} \KeywordTok{mean}\NormalTok{(Daily.Max.}\FloatTok{8.}\NormalTok{hour.Ozone.Concentration))}

\CommentTok{#Adding a new Date column that lists each month/year as a complete date including the first day of the month}
\NormalTok{GaringerOzone.monthly}\OperatorTok{$}\NormalTok{Date <-}\StringTok{ }\KeywordTok{seq}\NormalTok{(}\DataTypeTok{from =} \KeywordTok{as.Date}\NormalTok{(}\StringTok{"2010-01-01"}\NormalTok{), }\DataTypeTok{to =} 
                            \KeywordTok{as.Date}\NormalTok{(}\StringTok{"2019-12-01"}\NormalTok{), }\DataTypeTok{by =} \StringTok{"month"}\NormalTok{)}

\CommentTok{# 10}
\CommentTok{#Creating a time series}
\NormalTok{GaringerOzone.monthly.ts <-}\StringTok{ }\KeywordTok{ts}\NormalTok{(}
\NormalTok{  GaringerOzone.monthly}\OperatorTok{$}\NormalTok{Mean.Ozone.Conc, }\DataTypeTok{frequency =} \DecValTok{12}\NormalTok{, }
                        \DataTypeTok{start =} \KeywordTok{c}\NormalTok{(}\DecValTok{2010}\NormalTok{, }\DecValTok{1}\NormalTok{, }\DecValTok{1}\NormalTok{), }\DataTypeTok{end =} \KeywordTok{c}\NormalTok{(}\DecValTok{2019}\NormalTok{, }\DecValTok{12}\NormalTok{, }\DecValTok{1}\NormalTok{))}

\CommentTok{# 11}
\CommentTok{#Running a time series analysis (smk)}
\NormalTok{GaringerOzone.monthly.trend <-}\StringTok{ }\KeywordTok{smk.test}\NormalTok{(GaringerOzone.monthly.ts)}
\KeywordTok{summary}\NormalTok{(GaringerOzone.monthly.trend)}
\end{Highlighting}
\end{Shaded}

\begin{verbatim}
## 
##  Seasonal Mann-Kendall trend test (Hirsch-Slack test)
## 
## data: GaringerOzone.monthly.ts
## alternative hypothesis: two.sided
## 
## Statistics for individual seasons
## 
## H0
##                      S varS    tau      z Pr(>|z|)  
## Season 1:   S = 0   15  125  0.333  1.252  0.21050  
## Season 2:   S = 0   -1  125 -0.022  0.000  1.00000  
## Season 3:   S = 0   -4  124 -0.090 -0.269  0.78762  
## Season 4:   S = 0  -17  125 -0.378 -1.431  0.15241  
## Season 5:   S = 0  -15  125 -0.333 -1.252  0.21050  
## Season 6:   S = 0  -17  125 -0.378 -1.431  0.15241  
## Season 7:   S = 0  -11  125 -0.244 -0.894  0.37109  
## Season 8:   S = 0   -7  125 -0.156 -0.537  0.59151  
## Season 9:   S = 0   -5  125 -0.111 -0.358  0.72051  
## Season 10:   S = 0 -13  125 -0.289 -1.073  0.28313  
## Season 11:   S = 0 -13  125 -0.289 -1.073  0.28313  
## Season 12:   S = 0  11  125  0.244  0.894  0.37109  
## ---
## Signif. codes:  0 '***' 0.001 '**' 0.01 '*' 0.05 '.' 0.1 ' ' 1
\end{verbatim}

\begin{Shaded}
\begin{Highlighting}[]
\CommentTok{# 12}
\KeywordTok{sea.sens.slope}\NormalTok{(GaringerOzone.monthly.ts)}
\end{Highlighting}
\end{Shaded}

\begin{verbatim}
## [1] -0.0002044163
\end{verbatim}

\begin{Shaded}
\begin{Highlighting}[]
\CommentTok{# 13}
\KeywordTok{ggplot}\NormalTok{(}\DataTypeTok{data =}\NormalTok{ GaringerOzone.monthly, }
       \KeywordTok{aes}\NormalTok{(}\DataTypeTok{x =}\NormalTok{ Date, }\DataTypeTok{y =}\NormalTok{ Mean.Ozone.Conc)) }\OperatorTok{+}
\StringTok{  }\KeywordTok{geom_line}\NormalTok{() }\OperatorTok{+}
\StringTok{  }\KeywordTok{geom_point}\NormalTok{() }\OperatorTok{+}
\StringTok{  }\KeywordTok{labs}\NormalTok{(}\DataTypeTok{y =} \StringTok{"Mean Max 8 hour Ozone Conc (ppm)"}\NormalTok{)}
\end{Highlighting}
\end{Shaded}

\includegraphics{A08_TimeSeries_files/figure-latex/unnamed-chunk-4-1.pdf}

\begin{enumerate}
\def\labelenumi{\arabic{enumi}.}
\setcounter{enumi}{13}
\tightlist
\item
  To accompany your graph, summarize your results in context of the
  research question. Include output from the statistical test in
  parentheses at the end of your sentence. Feel free to use multiple
  sentences in your interpretation.
\end{enumerate}

\begin{quote}
Answer:
\end{quote}


\end{document}
