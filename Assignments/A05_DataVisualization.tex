\documentclass[]{article}
\usepackage{lmodern}
\usepackage{amssymb,amsmath}
\usepackage{ifxetex,ifluatex}
\usepackage{fixltx2e} % provides \textsubscript
\ifnum 0\ifxetex 1\fi\ifluatex 1\fi=0 % if pdftex
  \usepackage[T1]{fontenc}
  \usepackage[utf8]{inputenc}
\else % if luatex or xelatex
  \ifxetex
    \usepackage{mathspec}
  \else
    \usepackage{fontspec}
  \fi
  \defaultfontfeatures{Ligatures=TeX,Scale=MatchLowercase}
\fi
% use upquote if available, for straight quotes in verbatim environments
\IfFileExists{upquote.sty}{\usepackage{upquote}}{}
% use microtype if available
\IfFileExists{microtype.sty}{%
\usepackage{microtype}
\UseMicrotypeSet[protrusion]{basicmath} % disable protrusion for tt fonts
}{}
\usepackage[margin=2.54cm]{geometry}
\usepackage{hyperref}
\hypersetup{unicode=true,
            pdftitle={Assignment 5: Data Visualization},
            pdfauthor={Claire Mullaney},
            pdfborder={0 0 0},
            breaklinks=true}
\urlstyle{same}  % don't use monospace font for urls
\usepackage{color}
\usepackage{fancyvrb}
\newcommand{\VerbBar}{|}
\newcommand{\VERB}{\Verb[commandchars=\\\{\}]}
\DefineVerbatimEnvironment{Highlighting}{Verbatim}{commandchars=\\\{\}}
% Add ',fontsize=\small' for more characters per line
\usepackage{framed}
\definecolor{shadecolor}{RGB}{248,248,248}
\newenvironment{Shaded}{\begin{snugshade}}{\end{snugshade}}
\newcommand{\AlertTok}[1]{\textcolor[rgb]{0.94,0.16,0.16}{#1}}
\newcommand{\AnnotationTok}[1]{\textcolor[rgb]{0.56,0.35,0.01}{\textbf{\textit{#1}}}}
\newcommand{\AttributeTok}[1]{\textcolor[rgb]{0.77,0.63,0.00}{#1}}
\newcommand{\BaseNTok}[1]{\textcolor[rgb]{0.00,0.00,0.81}{#1}}
\newcommand{\BuiltInTok}[1]{#1}
\newcommand{\CharTok}[1]{\textcolor[rgb]{0.31,0.60,0.02}{#1}}
\newcommand{\CommentTok}[1]{\textcolor[rgb]{0.56,0.35,0.01}{\textit{#1}}}
\newcommand{\CommentVarTok}[1]{\textcolor[rgb]{0.56,0.35,0.01}{\textbf{\textit{#1}}}}
\newcommand{\ConstantTok}[1]{\textcolor[rgb]{0.00,0.00,0.00}{#1}}
\newcommand{\ControlFlowTok}[1]{\textcolor[rgb]{0.13,0.29,0.53}{\textbf{#1}}}
\newcommand{\DataTypeTok}[1]{\textcolor[rgb]{0.13,0.29,0.53}{#1}}
\newcommand{\DecValTok}[1]{\textcolor[rgb]{0.00,0.00,0.81}{#1}}
\newcommand{\DocumentationTok}[1]{\textcolor[rgb]{0.56,0.35,0.01}{\textbf{\textit{#1}}}}
\newcommand{\ErrorTok}[1]{\textcolor[rgb]{0.64,0.00,0.00}{\textbf{#1}}}
\newcommand{\ExtensionTok}[1]{#1}
\newcommand{\FloatTok}[1]{\textcolor[rgb]{0.00,0.00,0.81}{#1}}
\newcommand{\FunctionTok}[1]{\textcolor[rgb]{0.00,0.00,0.00}{#1}}
\newcommand{\ImportTok}[1]{#1}
\newcommand{\InformationTok}[1]{\textcolor[rgb]{0.56,0.35,0.01}{\textbf{\textit{#1}}}}
\newcommand{\KeywordTok}[1]{\textcolor[rgb]{0.13,0.29,0.53}{\textbf{#1}}}
\newcommand{\NormalTok}[1]{#1}
\newcommand{\OperatorTok}[1]{\textcolor[rgb]{0.81,0.36,0.00}{\textbf{#1}}}
\newcommand{\OtherTok}[1]{\textcolor[rgb]{0.56,0.35,0.01}{#1}}
\newcommand{\PreprocessorTok}[1]{\textcolor[rgb]{0.56,0.35,0.01}{\textit{#1}}}
\newcommand{\RegionMarkerTok}[1]{#1}
\newcommand{\SpecialCharTok}[1]{\textcolor[rgb]{0.00,0.00,0.00}{#1}}
\newcommand{\SpecialStringTok}[1]{\textcolor[rgb]{0.31,0.60,0.02}{#1}}
\newcommand{\StringTok}[1]{\textcolor[rgb]{0.31,0.60,0.02}{#1}}
\newcommand{\VariableTok}[1]{\textcolor[rgb]{0.00,0.00,0.00}{#1}}
\newcommand{\VerbatimStringTok}[1]{\textcolor[rgb]{0.31,0.60,0.02}{#1}}
\newcommand{\WarningTok}[1]{\textcolor[rgb]{0.56,0.35,0.01}{\textbf{\textit{#1}}}}
\usepackage{graphicx,grffile}
\makeatletter
\def\maxwidth{\ifdim\Gin@nat@width>\linewidth\linewidth\else\Gin@nat@width\fi}
\def\maxheight{\ifdim\Gin@nat@height>\textheight\textheight\else\Gin@nat@height\fi}
\makeatother
% Scale images if necessary, so that they will not overflow the page
% margins by default, and it is still possible to overwrite the defaults
% using explicit options in \includegraphics[width, height, ...]{}
\setkeys{Gin}{width=\maxwidth,height=\maxheight,keepaspectratio}
\IfFileExists{parskip.sty}{%
\usepackage{parskip}
}{% else
\setlength{\parindent}{0pt}
\setlength{\parskip}{6pt plus 2pt minus 1pt}
}
\setlength{\emergencystretch}{3em}  % prevent overfull lines
\providecommand{\tightlist}{%
  \setlength{\itemsep}{0pt}\setlength{\parskip}{0pt}}
\setcounter{secnumdepth}{0}
% Redefines (sub)paragraphs to behave more like sections
\ifx\paragraph\undefined\else
\let\oldparagraph\paragraph
\renewcommand{\paragraph}[1]{\oldparagraph{#1}\mbox{}}
\fi
\ifx\subparagraph\undefined\else
\let\oldsubparagraph\subparagraph
\renewcommand{\subparagraph}[1]{\oldsubparagraph{#1}\mbox{}}
\fi

%%% Use protect on footnotes to avoid problems with footnotes in titles
\let\rmarkdownfootnote\footnote%
\def\footnote{\protect\rmarkdownfootnote}

%%% Change title format to be more compact
\usepackage{titling}

% Create subtitle command for use in maketitle
\providecommand{\subtitle}[1]{
  \posttitle{
    \begin{center}\large#1\end{center}
    }
}

\setlength{\droptitle}{-2em}

  \title{Assignment 5: Data Visualization}
    \pretitle{\vspace{\droptitle}\centering\huge}
  \posttitle{\par}
    \author{Claire Mullaney}
    \preauthor{\centering\large\emph}
  \postauthor{\par}
    \date{}
    \predate{}\postdate{}
  

\begin{document}
\maketitle

\hypertarget{overview}{%
\subsection{OVERVIEW}\label{overview}}

This exercise accompanies the lessons in Environmental Data Analytics on
Data Visualization

\hypertarget{directions}{%
\subsection{Directions}\label{directions}}

\begin{enumerate}
\def\labelenumi{\arabic{enumi}.}
\tightlist
\item
  Change ``Student Name'' on line 3 (above) with your name.
\item
  Work through the steps, \textbf{creating code and output} that fulfill
  each instruction.
\item
  Be sure to \textbf{answer the questions} in this assignment document.
\item
  When you have completed the assignment, \textbf{Knit} the text and
  code into a single PDF file.
\item
  After Knitting, submit the completed exercise (PDF file) to the
  dropbox in Sakai. Add your last name into the file name (e.g.,
  ``Salk\_A05\_DataVisualization.Rmd'') prior to submission.
\end{enumerate}

The completed exercise is due on Tuesday, February 11 at 1:00 pm.

\hypertarget{set-up-your-session}{%
\subsection{Set up your session}\label{set-up-your-session}}

\begin{enumerate}
\def\labelenumi{\arabic{enumi}.}
\item
  Set up your session. Verify your working directory and load the
  tidyverse and cowplot packages. Upload the NTL-LTER processed data
  files for nutrients and chemistry/physics for Peter and Paul Lakes
  (tidy and gathered) and the processed data file for the Niwot Ridge
  litter dataset.
\item
  Make sure R is reading dates as date format; if not, change the format
  to date.
\end{enumerate}

\begin{Shaded}
\begin{Highlighting}[]
\CommentTok{#1}
\CommentTok{#Verifying directory, loading packages, and uploading files}
\KeywordTok{getwd}\NormalTok{()}
\end{Highlighting}
\end{Shaded}

\begin{verbatim}
## [1] "/Users/clairemullaney/Desktop/ENV 872/Environmental_Data_Analytics_2020"
\end{verbatim}

\begin{Shaded}
\begin{Highlighting}[]
\KeywordTok{library}\NormalTok{(ggplot2)}
\KeywordTok{library}\NormalTok{(viridis)}
\KeywordTok{library}\NormalTok{(RColorBrewer)}
\KeywordTok{library}\NormalTok{(tidyverse)}
\KeywordTok{library}\NormalTok{(cowplot)}


\NormalTok{Peter_Paul_Nutrients <-}\StringTok{ }
\StringTok{  }\KeywordTok{read.csv}\NormalTok{(}\StringTok{"./Data/Processed/NTL-LTER_Lake_Chemistry_Nutrients_PeterPaul_Processed.csv"}\NormalTok{)}
\NormalTok{Peter_Paul_Nutrients_Gathered <-}\StringTok{ }
\StringTok{  }\KeywordTok{read.csv}\NormalTok{(}\StringTok{"./Data/Processed/NTL-LTER_Lake_Nutrients_PeterPaulGathered_Processed.csv"}\NormalTok{)}
\NormalTok{Niwot_Litter <-}\StringTok{ }\KeywordTok{read.csv}\NormalTok{(}\StringTok{"./Data/Processed/NEON_NIWO_Litter_mass_trap_Processed.csv"}\NormalTok{)}

\CommentTok{#2}
\CommentTok{#Looking at column names, checking the class of each date column, and}
\CommentTok{#converting those date columns to dates}

\CommentTok{###Peter_Paul_Nutrients}
\KeywordTok{colnames}\NormalTok{(Peter_Paul_Nutrients)}
\end{Highlighting}
\end{Shaded}

\begin{verbatim}
##  [1] "lakename"        "year4"           "daynum"         
##  [4] "month"           "sampledate"      "depth"          
##  [7] "temperature_C"   "dissolvedOxygen" "irradianceWater"
## [10] "irradianceDeck"  "tn_ug"           "tp_ug"          
## [13] "nh34"            "no23"            "po4"
\end{verbatim}

\begin{Shaded}
\begin{Highlighting}[]
\KeywordTok{class}\NormalTok{(Peter_Paul_Nutrients}\OperatorTok{$}\NormalTok{sampledate)}
\end{Highlighting}
\end{Shaded}

\begin{verbatim}
## [1] "factor"
\end{verbatim}

\begin{Shaded}
\begin{Highlighting}[]
\NormalTok{Peter_Paul_Nutrients}\OperatorTok{$}\NormalTok{sampledate <-}\StringTok{ }
\StringTok{  }\KeywordTok{as.Date}\NormalTok{(Peter_Paul_Nutrients}\OperatorTok{$}\NormalTok{sampledate, }
          \DataTypeTok{format =} \StringTok{"%Y-%m-%d"}\NormalTok{)}

\KeywordTok{class}\NormalTok{(Peter_Paul_Nutrients}\OperatorTok{$}\NormalTok{sampledate)}
\end{Highlighting}
\end{Shaded}

\begin{verbatim}
## [1] "Date"
\end{verbatim}

\begin{Shaded}
\begin{Highlighting}[]
\CommentTok{###Peter_Paul_Nutrients_Gathered}
\KeywordTok{colnames}\NormalTok{(Peter_Paul_Nutrients_Gathered)}
\end{Highlighting}
\end{Shaded}

\begin{verbatim}
## [1] "lakename"      "year4"         "daynum"        "month"        
## [5] "sampledate"    "depth"         "nutrient"      "concentration"
\end{verbatim}

\begin{Shaded}
\begin{Highlighting}[]
\KeywordTok{class}\NormalTok{(Peter_Paul_Nutrients_Gathered}\OperatorTok{$}\NormalTok{sampledate)}
\end{Highlighting}
\end{Shaded}

\begin{verbatim}
## [1] "factor"
\end{verbatim}

\begin{Shaded}
\begin{Highlighting}[]
\NormalTok{Peter_Paul_Nutrients_Gathered}\OperatorTok{$}\NormalTok{sampledate <-}\StringTok{ }
\StringTok{  }\KeywordTok{as.Date}\NormalTok{(Peter_Paul_Nutrients_Gathered}\OperatorTok{$}\NormalTok{sampledate, }
          \DataTypeTok{format =} \StringTok{"%Y-%m-%d"}\NormalTok{)}

\KeywordTok{class}\NormalTok{(Peter_Paul_Nutrients_Gathered}\OperatorTok{$}\NormalTok{sampledate)}
\end{Highlighting}
\end{Shaded}

\begin{verbatim}
## [1] "Date"
\end{verbatim}

\begin{Shaded}
\begin{Highlighting}[]
\CommentTok{###Niwot_Litter}
\KeywordTok{colnames}\NormalTok{(Niwot_Litter)}
\end{Highlighting}
\end{Shaded}

\begin{verbatim}
##  [1] "plotID"           "trapID"           "collectDate"     
##  [4] "functionalGroup"  "dryMass"          "qaDryMass"       
##  [7] "subplotID"        "decimalLatitude"  "decimalLongitude"
## [10] "elevation"        "nlcdClass"        "plotType"        
## [13] "geodeticDatum"
\end{verbatim}

\begin{Shaded}
\begin{Highlighting}[]
\KeywordTok{class}\NormalTok{(Niwot_Litter}\OperatorTok{$}\NormalTok{collectDate)}
\end{Highlighting}
\end{Shaded}

\begin{verbatim}
## [1] "factor"
\end{verbatim}

\begin{Shaded}
\begin{Highlighting}[]
\NormalTok{Niwot_Litter}\OperatorTok{$}\NormalTok{collectDate <-}\StringTok{ }
\StringTok{  }\KeywordTok{as.Date}\NormalTok{(Niwot_Litter}\OperatorTok{$}\NormalTok{collectDate, }
          \DataTypeTok{format =} \StringTok{"%Y-%m-%d"}\NormalTok{)}

\KeywordTok{class}\NormalTok{(Niwot_Litter}\OperatorTok{$}\NormalTok{collectDate)}
\end{Highlighting}
\end{Shaded}

\begin{verbatim}
## [1] "Date"
\end{verbatim}

\hypertarget{define-your-theme}{%
\subsection{Define your theme}\label{define-your-theme}}

\begin{enumerate}
\def\labelenumi{\arabic{enumi}.}
\setcounter{enumi}{2}
\tightlist
\item
  Build a theme and set it as your default theme.
\end{enumerate}

\begin{Shaded}
\begin{Highlighting}[]
\CommentTok{#Defining a new theme}
\NormalTok{theme_}\DecValTok{5}\NormalTok{ <-}\StringTok{ }\KeywordTok{theme_classic}\NormalTok{(}\DataTypeTok{base_size =} \DecValTok{12}\NormalTok{) }\OperatorTok{+}
\StringTok{  }\KeywordTok{theme}\NormalTok{(}\DataTypeTok{axis.text =} \KeywordTok{element_text}\NormalTok{(}\DataTypeTok{color =} \StringTok{"black"}\NormalTok{), }
        \DataTypeTok{legend.position =} \StringTok{"right"}\NormalTok{)}

\CommentTok{#Setting new theme as the default theme}
\KeywordTok{theme_set}\NormalTok{(theme_}\DecValTok{5}\NormalTok{)}
\end{Highlighting}
\end{Shaded}

\hypertarget{create-graphs}{%
\subsection{Create graphs}\label{create-graphs}}

For numbers 4-7, create ggplot graphs and adjust aesthetics to follow
best practices for data visualization. Ensure your theme, color
palettes, axes, and additional aesthetics are edited accordingly.

\begin{enumerate}
\def\labelenumi{\arabic{enumi}.}
\setcounter{enumi}{3}
\tightlist
\item
  {[}NTL-LTER{]} Plot total phosphorus by phosphate, with separate
  aesthetics for Peter and Paul lakes. Add a line of best fit and color
  it black. Adjust your axes to hide extreme values.
\end{enumerate}

\begin{Shaded}
\begin{Highlighting}[]
\NormalTok{tp_vs_po4 <-}\StringTok{ }\KeywordTok{ggplot}\NormalTok{(Peter_Paul_Nutrients, }\KeywordTok{aes}\NormalTok{(}\DataTypeTok{y =}\NormalTok{ tp_ug, }\DataTypeTok{x =}\NormalTok{ po4, }\DataTypeTok{color =}\NormalTok{ lakename)) }\OperatorTok{+}
\StringTok{  }\KeywordTok{geom_point}\NormalTok{() }\OperatorTok{+}
\StringTok{  }\KeywordTok{geom_smooth}\NormalTok{(}\DataTypeTok{method=}\NormalTok{lm, }\DataTypeTok{se =} \OtherTok{FALSE}\NormalTok{, }
              \DataTypeTok{color =} \StringTok{"black"}\NormalTok{, }\DataTypeTok{size =} \FloatTok{0.5}\NormalTok{) }\OperatorTok{+}
\StringTok{  }\KeywordTok{xlim}\NormalTok{(}\DecValTok{0}\NormalTok{, }\DecValTok{50}\NormalTok{) }\OperatorTok{+}
\StringTok{  }\KeywordTok{ylim}\NormalTok{(}\DecValTok{0}\NormalTok{, }\DecValTok{150}\NormalTok{) }\OperatorTok{+}
\StringTok{  }\KeywordTok{labs}\NormalTok{(}\DataTypeTok{y =} \KeywordTok{expression}\NormalTok{(}\KeywordTok{paste}\NormalTok{(}\StringTok{"TP ( "}\NormalTok{, mu, }\StringTok{"g/L)"}\NormalTok{)), }
       \DataTypeTok{x =} \KeywordTok{expression}\NormalTok{(}\KeywordTok{paste}\NormalTok{(}\StringTok{"PO"}\NormalTok{[}\DecValTok{4}\NormalTok{]}\OperatorTok{*}\StringTok{ " ("}\NormalTok{, mu, }\StringTok{"g/L)"}\NormalTok{)), }
       \DataTypeTok{color =} \StringTok{"Lake Name"}\NormalTok{) }\OperatorTok{+}
\StringTok{  }\KeywordTok{scale_color_viridis}\NormalTok{(}\DataTypeTok{discrete =} \OtherTok{TRUE}\NormalTok{)}

\KeywordTok{print}\NormalTok{(tp_vs_po4)}
\end{Highlighting}
\end{Shaded}

\includegraphics{A05_DataVisualization_files/figure-latex/unnamed-chunk-4-1.pdf}

\begin{enumerate}
\def\labelenumi{\arabic{enumi}.}
\setcounter{enumi}{4}
\tightlist
\item
  {[}NTL-LTER{]} Make three separate boxplots of (a) temperature, (b)
  TP, and (c) TN, with month as the x axis and lake as a color
  aesthetic. Then, create a cowplot that combines the three graphs. Make
  sure that only one legend is present and that graph axes are aligned.
\end{enumerate}

\begin{Shaded}
\begin{Highlighting}[]
\CommentTok{#Constructing boxplots}

\NormalTok{temp_box <-}\StringTok{ }\KeywordTok{ggplot}\NormalTok{(Peter_Paul_Nutrients, }
                   \KeywordTok{aes}\NormalTok{(}\DataTypeTok{y =}\NormalTok{ temperature_C, }\DataTypeTok{x =} \KeywordTok{as.factor}\NormalTok{(month), }
                       \DataTypeTok{color =}\NormalTok{ lakename)) }\OperatorTok{+}
\StringTok{  }\KeywordTok{geom_boxplot}\NormalTok{() }\OperatorTok{+}\StringTok{ }
\StringTok{  }\KeywordTok{labs}\NormalTok{(}\DataTypeTok{y =} \KeywordTok{expression}\NormalTok{(}\StringTok{"Temp "}\NormalTok{ (degree}\OperatorTok{~}\NormalTok{C)), }\DataTypeTok{x =} \StringTok{"Month"}\NormalTok{, }\DataTypeTok{color =} \StringTok{"Lake Name"}\NormalTok{) }\OperatorTok{+}
\StringTok{  }\KeywordTok{scale_color_manual}\NormalTok{(}\DataTypeTok{values =} \KeywordTok{c}\NormalTok{(}\StringTok{"#d7191c"}\NormalTok{, }\StringTok{"#2c7bb6"}\NormalTok{))}

\KeywordTok{print}\NormalTok{(temp_box)}
\end{Highlighting}
\end{Shaded}

\includegraphics{A05_DataVisualization_files/figure-latex/unnamed-chunk-5-1.pdf}

\begin{Shaded}
\begin{Highlighting}[]
\CommentTok{##########}
\NormalTok{tp_box <-}\StringTok{ }\KeywordTok{ggplot}\NormalTok{(Peter_Paul_Nutrients, }
                 \KeywordTok{aes}\NormalTok{(}\DataTypeTok{y =}\NormalTok{ tp_ug, }\DataTypeTok{x =} \KeywordTok{as.factor}\NormalTok{(month), }
                     \DataTypeTok{color =}\NormalTok{ lakename)) }\OperatorTok{+}
\StringTok{         }\KeywordTok{geom_boxplot}\NormalTok{() }\OperatorTok{+}\StringTok{ }
\StringTok{  }\KeywordTok{labs}\NormalTok{(}\DataTypeTok{y =} \KeywordTok{expression}\NormalTok{(}\KeywordTok{paste}\NormalTok{(}\StringTok{"TP ( "}\NormalTok{, mu, }\StringTok{"g/L)"}\NormalTok{)), }
       \DataTypeTok{x =} \StringTok{"Month"}\NormalTok{, }\DataTypeTok{color =} \StringTok{"Lake Name"}\NormalTok{) }\OperatorTok{+}
\StringTok{  }\KeywordTok{scale_color_manual}\NormalTok{(}\DataTypeTok{values =} \KeywordTok{c}\NormalTok{(}\StringTok{"#d7191c"}\NormalTok{, }\StringTok{"#2c7bb6"}\NormalTok{))}
  
\KeywordTok{print}\NormalTok{(tp_box)}
\end{Highlighting}
\end{Shaded}

\includegraphics{A05_DataVisualization_files/figure-latex/unnamed-chunk-5-2.pdf}

\begin{Shaded}
\begin{Highlighting}[]
\CommentTok{##########}
\NormalTok{tn_box <-}\StringTok{ }\KeywordTok{ggplot}\NormalTok{(Peter_Paul_Nutrients, }
                 \KeywordTok{aes}\NormalTok{(}\DataTypeTok{y =}\NormalTok{ tn_ug, }\DataTypeTok{x =} \KeywordTok{as.factor}\NormalTok{(month), }
                     \DataTypeTok{color =}\NormalTok{ lakename)) }\OperatorTok{+}
\StringTok{         }\KeywordTok{geom_boxplot}\NormalTok{() }\OperatorTok{+}
\StringTok{    }\KeywordTok{labs}\NormalTok{(}\DataTypeTok{y =} \KeywordTok{expression}\NormalTok{(}\KeywordTok{paste}\NormalTok{(}\StringTok{"TN ( "}\NormalTok{, mu, }\StringTok{"g/L)"}\NormalTok{)), }\DataTypeTok{x =} \StringTok{"Month"}\NormalTok{,}
         \DataTypeTok{color =} \StringTok{"Lake Name"}\NormalTok{) }\OperatorTok{+}
\StringTok{  }\KeywordTok{scale_color_manual}\NormalTok{(}\DataTypeTok{values =} \KeywordTok{c}\NormalTok{(}\StringTok{"#d7191c"}\NormalTok{, }\StringTok{"#2c7bb6"}\NormalTok{))}

\KeywordTok{print}\NormalTok{(tn_box)}
\end{Highlighting}
\end{Shaded}

\includegraphics{A05_DataVisualization_files/figure-latex/unnamed-chunk-5-3.pdf}

\begin{Shaded}
\begin{Highlighting}[]
\CommentTok{#Create a cowplot that combines the three graphs. Make sure that only}
\CommentTok{#one legend is present and that graph axes are aligned.}

\NormalTok{temp_tp_tn <-}\StringTok{ }\KeywordTok{plot_grid}\NormalTok{(temp_box }\OperatorTok{+}\StringTok{ }\KeywordTok{theme}\NormalTok{(}\DataTypeTok{legend.position=}\StringTok{"none"}\NormalTok{),}
\NormalTok{                        tp_box }\OperatorTok{+}\StringTok{ }\KeywordTok{theme}\NormalTok{(}\DataTypeTok{legend.position=}\StringTok{"none"}\NormalTok{), }
\NormalTok{                        tn_box }\OperatorTok{+}\StringTok{ }\KeywordTok{theme}\NormalTok{(}\DataTypeTok{legend.position=}\StringTok{"none"}\NormalTok{), }
                        \DataTypeTok{nrow =} \DecValTok{3}\NormalTok{, }\DataTypeTok{align =} \StringTok{"hv"}\NormalTok{)}

\NormalTok{legend <-}\StringTok{ }\KeywordTok{get_legend}\NormalTok{(temp_box }\OperatorTok{+}\StringTok{ }
\StringTok{                       }\KeywordTok{guides}\NormalTok{(}\DataTypeTok{color =} \KeywordTok{guide_legend}\NormalTok{(}\DataTypeTok{nrow =} \DecValTok{1}\NormalTok{)) }\OperatorTok{+}\StringTok{ }
\StringTok{                       }\KeywordTok{theme}\NormalTok{(}\DataTypeTok{legend.position =} \StringTok{"bottom"}\NormalTok{))}

\KeywordTok{plot_grid}\NormalTok{(temp_tp_tn, legend, }\DataTypeTok{ncol =} \DecValTok{1}\NormalTok{, }\DataTypeTok{rel_heights =} \KeywordTok{c}\NormalTok{(}\DecValTok{1}\NormalTok{, }\FloatTok{.1}\NormalTok{))}
\end{Highlighting}
\end{Shaded}

\includegraphics{A05_DataVisualization_files/figure-latex/unnamed-chunk-5-4.pdf}

Question: What do you observe about the variables of interest over
seasons and between lakes?

\begin{quote}
Answer: Median temperature and temperature interquartile ranges (IQRs)
increase from May through August and decrease from September through
November. Peter Lake consistently has lower median temperatures, and
often wider temperature IQRs, than Paul Lake (except in the month of
October). The total amount of nitrogen in the lakes follows a similar
pattern, with the median amount slightly increasing for both lakes from
May through August and starting to decrease in September. Peter Lake
often has larger median amounts of nitrogen, larger IQRs, and more
positive skew than Paul Lake. Unlike temperature and total nitrogen, the
median total amount of phosphorus in the lakes appears to increase from
May all the way through September, the last month for which there is
data. This increase is more prominent for Peter Lake, which also has
higher median amounts of total phosphorus and larger IQRs. The
distributions of total amounts of phosphorus for Paul Lake appear to be
more positively skewed for any given month. For both Peter and Paul
lake, there are many positive outliers in the distributions for both
total nitrogen and total phosphorus in each month.
\end{quote}

\begin{enumerate}
\def\labelenumi{\arabic{enumi}.}
\setcounter{enumi}{5}
\item
  {[}Niwot Ridge{]} Plot a subset of the litter dataset by displaying
  only the ``Needles'' functional group. Plot the dry mass of needle
  litter by date and separate by NLCD class with a color aesthetic. (no
  need to adjust the name of each land use)
\item
  {[}Niwot Ridge{]} Now, plot the same plot but with NLCD classes
  separated into three facets rather than separated by color.
\end{enumerate}

\begin{Shaded}
\begin{Highlighting}[]
\CommentTok{#6}
\CommentTok{#Color-coded graph}
\NormalTok{drymass_vs_date <-}\StringTok{ }\KeywordTok{ggplot}\NormalTok{(}\KeywordTok{subset}\NormalTok{(Niwot_Litter, }
\NormalTok{                                 functionalGroup }\OperatorTok{==}\StringTok{ "Needles"}\NormalTok{)) }\OperatorTok{+}
\StringTok{  }\KeywordTok{aes}\NormalTok{(}\DataTypeTok{y =}\NormalTok{ dryMass, }\DataTypeTok{x =}\NormalTok{ collectDate, }\DataTypeTok{color =}\NormalTok{ nlcdClass) }\OperatorTok{+}
\StringTok{  }\KeywordTok{geom_point}\NormalTok{() }\OperatorTok{+}
\StringTok{  }\KeywordTok{labs}\NormalTok{(}\DataTypeTok{y =} \StringTok{"Dry Mass (g)"}\NormalTok{, }\DataTypeTok{x =} \StringTok{"Collection Date"}\NormalTok{, }\DataTypeTok{color =} \StringTok{"NLCD Class"}\NormalTok{) }\OperatorTok{+}
\StringTok{   }\KeywordTok{scale_color_viridis}\NormalTok{(}\DataTypeTok{discrete =} \OtherTok{TRUE}\NormalTok{)}

\KeywordTok{print}\NormalTok{(drymass_vs_date)}
\end{Highlighting}
\end{Shaded}

\includegraphics{A05_DataVisualization_files/figure-latex/unnamed-chunk-6-1.pdf}

\begin{Shaded}
\begin{Highlighting}[]
\CommentTok{#7}
\CommentTok{#Faceted graph}

\NormalTok{drymass_vs_date_fac <-}\StringTok{ }\KeywordTok{ggplot}\NormalTok{(}\KeywordTok{subset}\NormalTok{(Niwot_Litter, }
\NormalTok{                                     functionalGroup }\OperatorTok{==}
\StringTok{                                       "Needles"}\NormalTok{)) }\OperatorTok{+}
\StringTok{  }\KeywordTok{aes}\NormalTok{(}\DataTypeTok{y =}\NormalTok{ dryMass, }\DataTypeTok{x =}\NormalTok{ collectDate) }\OperatorTok{+}
\StringTok{  }\KeywordTok{geom_point}\NormalTok{() }\OperatorTok{+}
\StringTok{  }\KeywordTok{facet_wrap}\NormalTok{(}\KeywordTok{vars}\NormalTok{(nlcdClass), }\DataTypeTok{nrow =} \DecValTok{3}\NormalTok{) }\OperatorTok{+}
\StringTok{  }\KeywordTok{labs}\NormalTok{(}\DataTypeTok{y =} \StringTok{"Dry Mass (g)"}\NormalTok{, }\DataTypeTok{x =} \StringTok{"Collection Date"}\NormalTok{)}

\KeywordTok{print}\NormalTok{(drymass_vs_date_fac)}
\end{Highlighting}
\end{Shaded}

\includegraphics{A05_DataVisualization_files/figure-latex/unnamed-chunk-6-2.pdf}

Question: Which of these plots (6 vs.~7) do you think is more effective,
and why?

\begin{quote}
Answer: I think plot 7 is more effective; when facet\_wrap is used to
create three individual graphs (one for each NLCD class), each in its
own row, it is easy to see the dry masses of each NLCD type within any
given year. These dry masses can be efficiently compared both across
classes (by looking at a vertical segment of all three plots) and across
years (by looking at each individual graph). In plot 6, even though the
NLCD classes are separated by color, it is harder to collectively see
the dry masses of each individual NLCD class for any given date range.
While this difficulty does not result in comparisons across classes
within a given date range being terribly cumbersome (although overlap in
some data points does make these comparisons a bit more challenging than
with plot 7), it does make examining changes in dry mass for one
individual class across all years much more difficult in plot 6 than
plot 7.
\end{quote}


\end{document}
