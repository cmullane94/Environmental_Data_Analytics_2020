\documentclass[]{article}
\usepackage{lmodern}
\usepackage{amssymb,amsmath}
\usepackage{ifxetex,ifluatex}
\usepackage{fixltx2e} % provides \textsubscript
\ifnum 0\ifxetex 1\fi\ifluatex 1\fi=0 % if pdftex
  \usepackage[T1]{fontenc}
  \usepackage[utf8]{inputenc}
\else % if luatex or xelatex
  \ifxetex
    \usepackage{mathspec}
  \else
    \usepackage{fontspec}
  \fi
  \defaultfontfeatures{Ligatures=TeX,Scale=MatchLowercase}
\fi
% use upquote if available, for straight quotes in verbatim environments
\IfFileExists{upquote.sty}{\usepackage{upquote}}{}
% use microtype if available
\IfFileExists{microtype.sty}{%
\usepackage{microtype}
\UseMicrotypeSet[protrusion]{basicmath} % disable protrusion for tt fonts
}{}
\usepackage[margin=2.54cm]{geometry}
\usepackage{hyperref}
\hypersetup{unicode=true,
            pdftitle={Assignment 10: Data Scraping},
            pdfauthor={Claire Mullaney},
            pdfborder={0 0 0},
            breaklinks=true}
\urlstyle{same}  % don't use monospace font for urls
\usepackage{color}
\usepackage{fancyvrb}
\newcommand{\VerbBar}{|}
\newcommand{\VERB}{\Verb[commandchars=\\\{\}]}
\DefineVerbatimEnvironment{Highlighting}{Verbatim}{commandchars=\\\{\}}
% Add ',fontsize=\small' for more characters per line
\usepackage{framed}
\definecolor{shadecolor}{RGB}{248,248,248}
\newenvironment{Shaded}{\begin{snugshade}}{\end{snugshade}}
\newcommand{\AlertTok}[1]{\textcolor[rgb]{0.94,0.16,0.16}{#1}}
\newcommand{\AnnotationTok}[1]{\textcolor[rgb]{0.56,0.35,0.01}{\textbf{\textit{#1}}}}
\newcommand{\AttributeTok}[1]{\textcolor[rgb]{0.77,0.63,0.00}{#1}}
\newcommand{\BaseNTok}[1]{\textcolor[rgb]{0.00,0.00,0.81}{#1}}
\newcommand{\BuiltInTok}[1]{#1}
\newcommand{\CharTok}[1]{\textcolor[rgb]{0.31,0.60,0.02}{#1}}
\newcommand{\CommentTok}[1]{\textcolor[rgb]{0.56,0.35,0.01}{\textit{#1}}}
\newcommand{\CommentVarTok}[1]{\textcolor[rgb]{0.56,0.35,0.01}{\textbf{\textit{#1}}}}
\newcommand{\ConstantTok}[1]{\textcolor[rgb]{0.00,0.00,0.00}{#1}}
\newcommand{\ControlFlowTok}[1]{\textcolor[rgb]{0.13,0.29,0.53}{\textbf{#1}}}
\newcommand{\DataTypeTok}[1]{\textcolor[rgb]{0.13,0.29,0.53}{#1}}
\newcommand{\DecValTok}[1]{\textcolor[rgb]{0.00,0.00,0.81}{#1}}
\newcommand{\DocumentationTok}[1]{\textcolor[rgb]{0.56,0.35,0.01}{\textbf{\textit{#1}}}}
\newcommand{\ErrorTok}[1]{\textcolor[rgb]{0.64,0.00,0.00}{\textbf{#1}}}
\newcommand{\ExtensionTok}[1]{#1}
\newcommand{\FloatTok}[1]{\textcolor[rgb]{0.00,0.00,0.81}{#1}}
\newcommand{\FunctionTok}[1]{\textcolor[rgb]{0.00,0.00,0.00}{#1}}
\newcommand{\ImportTok}[1]{#1}
\newcommand{\InformationTok}[1]{\textcolor[rgb]{0.56,0.35,0.01}{\textbf{\textit{#1}}}}
\newcommand{\KeywordTok}[1]{\textcolor[rgb]{0.13,0.29,0.53}{\textbf{#1}}}
\newcommand{\NormalTok}[1]{#1}
\newcommand{\OperatorTok}[1]{\textcolor[rgb]{0.81,0.36,0.00}{\textbf{#1}}}
\newcommand{\OtherTok}[1]{\textcolor[rgb]{0.56,0.35,0.01}{#1}}
\newcommand{\PreprocessorTok}[1]{\textcolor[rgb]{0.56,0.35,0.01}{\textit{#1}}}
\newcommand{\RegionMarkerTok}[1]{#1}
\newcommand{\SpecialCharTok}[1]{\textcolor[rgb]{0.00,0.00,0.00}{#1}}
\newcommand{\SpecialStringTok}[1]{\textcolor[rgb]{0.31,0.60,0.02}{#1}}
\newcommand{\StringTok}[1]{\textcolor[rgb]{0.31,0.60,0.02}{#1}}
\newcommand{\VariableTok}[1]{\textcolor[rgb]{0.00,0.00,0.00}{#1}}
\newcommand{\VerbatimStringTok}[1]{\textcolor[rgb]{0.31,0.60,0.02}{#1}}
\newcommand{\WarningTok}[1]{\textcolor[rgb]{0.56,0.35,0.01}{\textbf{\textit{#1}}}}
\usepackage{graphicx,grffile}
\makeatletter
\def\maxwidth{\ifdim\Gin@nat@width>\linewidth\linewidth\else\Gin@nat@width\fi}
\def\maxheight{\ifdim\Gin@nat@height>\textheight\textheight\else\Gin@nat@height\fi}
\makeatother
% Scale images if necessary, so that they will not overflow the page
% margins by default, and it is still possible to overwrite the defaults
% using explicit options in \includegraphics[width, height, ...]{}
\setkeys{Gin}{width=\maxwidth,height=\maxheight,keepaspectratio}
\IfFileExists{parskip.sty}{%
\usepackage{parskip}
}{% else
\setlength{\parindent}{0pt}
\setlength{\parskip}{6pt plus 2pt minus 1pt}
}
\setlength{\emergencystretch}{3em}  % prevent overfull lines
\providecommand{\tightlist}{%
  \setlength{\itemsep}{0pt}\setlength{\parskip}{0pt}}
\setcounter{secnumdepth}{0}
% Redefines (sub)paragraphs to behave more like sections
\ifx\paragraph\undefined\else
\let\oldparagraph\paragraph
\renewcommand{\paragraph}[1]{\oldparagraph{#1}\mbox{}}
\fi
\ifx\subparagraph\undefined\else
\let\oldsubparagraph\subparagraph
\renewcommand{\subparagraph}[1]{\oldsubparagraph{#1}\mbox{}}
\fi

%%% Use protect on footnotes to avoid problems with footnotes in titles
\let\rmarkdownfootnote\footnote%
\def\footnote{\protect\rmarkdownfootnote}

%%% Change title format to be more compact
\usepackage{titling}

% Create subtitle command for use in maketitle
\providecommand{\subtitle}[1]{
  \posttitle{
    \begin{center}\large#1\end{center}
    }
}

\setlength{\droptitle}{-2em}

  \title{Assignment 10: Data Scraping}
    \pretitle{\vspace{\droptitle}\centering\huge}
  \posttitle{\par}
    \author{Claire Mullaney}
    \preauthor{\centering\large\emph}
  \postauthor{\par}
    \date{}
    \predate{}\postdate{}
  

\begin{document}
\maketitle

\hypertarget{total-points}{%
\section{Total points:}\label{total-points}}

\hypertarget{overview}{%
\subsection{OVERVIEW}\label{overview}}

This exercise accompanies the lessons in Environmental Data Analytics on
time series analysis.

\hypertarget{directions}{%
\subsection{Directions}\label{directions}}

\begin{enumerate}
\def\labelenumi{\arabic{enumi}.}
\tightlist
\item
  Change ``Student Name'' on line 3 (above) with your name.
\item
  Work through the steps, \textbf{creating code and output} that fulfill
  each instruction.
\item
  Be sure to \textbf{answer the questions} in this assignment document.
\item
  When you have completed the assignment, \textbf{Knit} the text and
  code into a single PDF file.
\item
  After Knitting, submit the completed exercise (PDF file) to the
  dropbox in Sakai. Add your last name into the file name (e.g.,
  ``Salk\_A06\_GLMs\_Week1.Rmd'') prior to submission.
\end{enumerate}

The completed exercise is due on Tuesday, April 7 at 1:00 pm.

\hypertarget{set-up}{%
\subsection{Set up}\label{set-up}}

\begin{enumerate}
\def\labelenumi{\arabic{enumi}.}
\tightlist
\item
  Set up your session:
\end{enumerate}

\begin{itemize}
\tightlist
\item
  Check your working directory
\item
  Load the packages \texttt{tidyverse}, \texttt{rvest}, and any others
  you end up using.
\item
  Set your ggplot theme
\end{itemize}

\begin{Shaded}
\begin{Highlighting}[]
\KeywordTok{getwd}\NormalTok{()}
\end{Highlighting}
\end{Shaded}

\begin{verbatim}
## [1] "/Users/clairemullaney/Desktop/ENV 872/Environmental_Data_Analytics_2020"
\end{verbatim}

\begin{Shaded}
\begin{Highlighting}[]
\KeywordTok{library}\NormalTok{(tidyverse)}
\KeywordTok{library}\NormalTok{(rvest)}

\CommentTok{# Set theme}
\NormalTok{ggtheme <-}\StringTok{ }\KeywordTok{theme_classic}\NormalTok{(}\DataTypeTok{base_size =} \DecValTok{14}\NormalTok{) }\OperatorTok{+}\StringTok{ }
\StringTok{  }\KeywordTok{theme}\NormalTok{(}\DataTypeTok{axis.text =} \KeywordTok{element_text}\NormalTok{(}\DataTypeTok{color =} \StringTok{"black"}\NormalTok{),}
        \DataTypeTok{legend.position =} \StringTok{"right"}\NormalTok{)}
\KeywordTok{theme_set}\NormalTok{(ggtheme)}
\end{Highlighting}
\end{Shaded}

\begin{enumerate}
\def\labelenumi{\arabic{enumi}.}
\setcounter{enumi}{1}
\tightlist
\item
  Indicate the EPA impaired waters website
  (\url{https://www.epa.gov/nutrient-policy-data/waters-assessed-impaired-due-nutrient-related-causes})
  as the URL to be scraped.
\end{enumerate}

\begin{Shaded}
\begin{Highlighting}[]
\NormalTok{url <-}\StringTok{ "https://www.epa.gov/nutrient-policy-data/waters-assessed-impaired-due-nutrient-related-causes"}

\NormalTok{webpage <-}\StringTok{ }\KeywordTok{read_html}\NormalTok{(url)}
\end{Highlighting}
\end{Shaded}

\begin{enumerate}
\def\labelenumi{\arabic{enumi}.}
\setcounter{enumi}{2}
\tightlist
\item
  Scrape the Rivers table, with every column except year. Then, turn it
  into a data frame.
\end{enumerate}

\begin{Shaded}
\begin{Highlighting}[]
\NormalTok{State <-}\StringTok{ }\NormalTok{webpage }\OperatorTok
\StringTok{  }\KeywordTok{html_nodes}\NormalTok{(}\StringTok{"table:nth-child(8) td:nth-child(1)"}\NormalTok{) }\OperatorTok
\StringTok{  }\KeywordTok{html_text}\NormalTok{()}
\NormalTok{Rivers.Assessed.mi <-}\StringTok{ }\NormalTok{webpage }\OperatorTok
\StringTok{  }\KeywordTok{html_nodes}\NormalTok{(}\StringTok{"table:nth-child(8) td:nth-child(2)"}\NormalTok{) }\OperatorTok
\StringTok{  }\KeywordTok{html_text}\NormalTok{()}
\NormalTok{Rivers.Assessed.percent <-}\StringTok{ }\NormalTok{webpage }\OperatorTok
\StringTok{  }\KeywordTok{html_nodes}\NormalTok{(}\StringTok{"table:nth-child(8) td:nth-child(3)"}\NormalTok{) }\OperatorTok
\StringTok{  }\KeywordTok{html_text}\NormalTok{()}
\NormalTok{Rivers.Impaired.mi <-}\StringTok{ }\NormalTok{webpage }\OperatorTok
\StringTok{  }\KeywordTok{html_nodes}\NormalTok{(}\StringTok{"table:nth-child(8) td:nth-child(4)"}\NormalTok{) }\OperatorTok
\StringTok{  }\KeywordTok{html_text}\NormalTok{()}
\NormalTok{Rivers.Impaired.percent <-}\StringTok{ }\NormalTok{webpage }\OperatorTok
\StringTok{  }\KeywordTok{html_nodes}\NormalTok{(}\StringTok{"table:nth-child(8) td:nth-child(5)"}\NormalTok{) }\OperatorTok
\StringTok{  }\KeywordTok{html_text}\NormalTok{()}
\NormalTok{Rivers.Impaired.percent.TMDL <-}\StringTok{ }\NormalTok{webpage }\OperatorTok
\StringTok{  }\KeywordTok{html_nodes}\NormalTok{(}\StringTok{"table:nth-child(8) td:nth-child(6)"}\NormalTok{) }\OperatorTok
\StringTok{  }\KeywordTok{html_text}\NormalTok{()}

\NormalTok{Rivers <-}\StringTok{ }\KeywordTok{data.frame}\NormalTok{(State, Rivers.Assessed.mi, Rivers.Assessed.percent,}
\NormalTok{                     Rivers.Impaired.mi, Rivers.Impaired.percent, }
\NormalTok{                     Rivers.Impaired.percent.TMDL)}
\end{Highlighting}
\end{Shaded}

\begin{enumerate}
\def\labelenumi{\arabic{enumi}.}
\setcounter{enumi}{3}
\item
  Use \texttt{str\_replace} to remove non-numeric characters from the
  numeric columns.
\item
  Set the numeric columns to a numeric class and verify this using
  \texttt{str}.
\end{enumerate}

\begin{Shaded}
\begin{Highlighting}[]
\CommentTok{# 4}
\NormalTok{Rivers}\OperatorTok{$}\NormalTok{Rivers.Assessed.mi <-}\StringTok{ }\KeywordTok{str_replace}\NormalTok{(Rivers}\OperatorTok{$}\NormalTok{Rivers.Assessed.mi, }
                                          \DataTypeTok{pattern =} \StringTok{"([,])"}\NormalTok{, }\DataTypeTok{replacement =} \StringTok{""}\NormalTok{)}
\NormalTok{Rivers}\OperatorTok{$}\NormalTok{Rivers.Assessed.percent <-}\StringTok{ }\KeywordTok{str_replace}\NormalTok{(Rivers}\OperatorTok{$}\NormalTok{Rivers.Assessed.percent,}
                                              \DataTypeTok{pattern =} \StringTok{"([%])"}\NormalTok{, }\DataTypeTok{replacement =} \StringTok{""}\NormalTok{)}
\NormalTok{Rivers}\OperatorTok{$}\NormalTok{Rivers.Assessed.percent <-}\StringTok{ }\KeywordTok{str_replace}\NormalTok{(Rivers}\OperatorTok{$}\NormalTok{Rivers.Assessed.percent, }
                                              \DataTypeTok{pattern =} \StringTok{"([*])"}\NormalTok{, }\DataTypeTok{replacement =} \StringTok{""}\NormalTok{)}
\NormalTok{Rivers}\OperatorTok{$}\NormalTok{Rivers.Impaired.mi <-}\StringTok{ }\KeywordTok{str_replace}\NormalTok{(Rivers}\OperatorTok{$}\NormalTok{Rivers.Impaired.mi,}
                                          \DataTypeTok{pattern =} \StringTok{"([,])"}\NormalTok{, }\DataTypeTok{replacement =} \StringTok{""}\NormalTok{)}
\NormalTok{Rivers}\OperatorTok{$}\NormalTok{Rivers.Impaired.percent <-}\StringTok{ }\KeywordTok{str_replace}\NormalTok{(Rivers}\OperatorTok{$}\NormalTok{Rivers.Impaired.percent,}
                                              \DataTypeTok{pattern =} \StringTok{"([%])"}\NormalTok{, }\DataTypeTok{replacement =} \StringTok{""}\NormalTok{)}
\NormalTok{Rivers}\OperatorTok{$}\NormalTok{Rivers.Impaired.percent.TMDL <-}\StringTok{ }\KeywordTok{str_replace}\NormalTok{(Rivers}\OperatorTok{$}\NormalTok{Rivers.Impaired.percent.TMDL,}
                                                   \DataTypeTok{pattern =} \StringTok{"([%])"}\NormalTok{, }\DataTypeTok{replacement =} \StringTok{""}\NormalTok{)}
\NormalTok{Rivers}\OperatorTok{$}\NormalTok{Rivers.Impaired.percent.TMDL <-}\StringTok{ }\KeywordTok{str_replace}\NormalTok{(Rivers}\OperatorTok{$}\NormalTok{Rivers.Impaired.percent.TMDL,}
                                                   \DataTypeTok{pattern =} \StringTok{"([±])"}\NormalTok{, }\DataTypeTok{replacement =} \StringTok{""}\NormalTok{)}

\CommentTok{# 5}
\KeywordTok{str}\NormalTok{(Rivers)}
\end{Highlighting}
\end{Shaded}

\begin{verbatim}
## 'data.frame':    50 obs. of  6 variables:
##  $ State                       : Factor w/ 50 levels "Alabama","Alaska",..: 1 2 3 4 5 6 7 8 9 10 ...
##  $ Rivers.Assessed.mi          : chr  "10538" "602" "2764" "9979" ...
##  $ Rivers.Assessed.percent     : chr  "14" "0" "3" "11" ...
##  $ Rivers.Impaired.mi          : chr  "1146" "15" "144" "1440" ...
##  $ Rivers.Impaired.percent     : chr  "11" "2" "5" "14" ...
##  $ Rivers.Impaired.percent.TMDL: chr  "53" "100" "6" "2" ...
\end{verbatim}

\begin{Shaded}
\begin{Highlighting}[]
\NormalTok{Rivers}\OperatorTok{$}\NormalTok{Rivers.Assessed.mi <-}\StringTok{ }\KeywordTok{as.numeric}\NormalTok{(Rivers}\OperatorTok{$}\NormalTok{Rivers.Assessed.mi)}
\NormalTok{Rivers}\OperatorTok{$}\NormalTok{Rivers.Assessed.percent <-}\StringTok{ }\KeywordTok{as.numeric}\NormalTok{(Rivers}\OperatorTok{$}\NormalTok{Rivers.Assessed.percent)}
\NormalTok{Rivers}\OperatorTok{$}\NormalTok{Rivers.Impaired.mi <-}\StringTok{ }\KeywordTok{as.numeric}\NormalTok{(Rivers}\OperatorTok{$}\NormalTok{Rivers.Impaired.mi)}
\NormalTok{Rivers}\OperatorTok{$}\NormalTok{Rivers.Impaired.percent <-}\StringTok{ }\KeywordTok{as.numeric}\NormalTok{(Rivers}\OperatorTok{$}\NormalTok{Rivers.Impaired.percent)}
\NormalTok{Rivers}\OperatorTok{$}\NormalTok{Rivers.Impaired.percent.TMDL <-}\StringTok{ }\KeywordTok{as.numeric}\NormalTok{(Rivers}\OperatorTok{$}\NormalTok{Rivers.Impaired.percent.TMDL)}

\KeywordTok{str}\NormalTok{(Rivers)}
\end{Highlighting}
\end{Shaded}

\begin{verbatim}
## 'data.frame':    50 obs. of  6 variables:
##  $ State                       : Factor w/ 50 levels "Alabama","Alaska",..: 1 2 3 4 5 6 7 8 9 10 ...
##  $ Rivers.Assessed.mi          : num  10538 602 2764 9979 32803 ...
##  $ Rivers.Assessed.percent     : num  14 0 3 11 16 56 41 100 20 19 ...
##  $ Rivers.Impaired.mi          : num  1146 15 144 1440 13350 ...
##  $ Rivers.Impaired.percent     : num  11 2 5 14 41 0 0 88 53 9 ...
##  $ Rivers.Impaired.percent.TMDL: num  53 100 6 2 NA 14 73 37 NA 78 ...
\end{verbatim}

\begin{enumerate}
\def\labelenumi{\arabic{enumi}.}
\setcounter{enumi}{5}
\tightlist
\item
  Scrape the Lakes table, with every column except year. Then, turn it
  into a data frame.
\end{enumerate}

\begin{Shaded}
\begin{Highlighting}[]
\NormalTok{State <-}\StringTok{ }\NormalTok{webpage }\OperatorTok
\StringTok{  }\KeywordTok{html_nodes}\NormalTok{(}\StringTok{"table:nth-child(14) td:nth-child(1)"}\NormalTok{) }\OperatorTok
\StringTok{  }\KeywordTok{html_text}\NormalTok{()}
\NormalTok{Lakes.Assessed.acres <-}\StringTok{ }\NormalTok{webpage }\OperatorTok
\StringTok{  }\KeywordTok{html_nodes}\NormalTok{(}\StringTok{"table:nth-child(14) td:nth-child(2)"}\NormalTok{) }\OperatorTok
\StringTok{  }\KeywordTok{html_text}\NormalTok{()}
\NormalTok{Lakes.Assessed.percent <-}\StringTok{ }\NormalTok{webpage }\OperatorTok
\StringTok{  }\KeywordTok{html_nodes}\NormalTok{(}\StringTok{"table:nth-child(14) td:nth-child(3)"}\NormalTok{) }\OperatorTok
\StringTok{  }\KeywordTok{html_text}\NormalTok{()}
\NormalTok{Lakes.Impaired.acres <-}\StringTok{ }\NormalTok{webpage }\OperatorTok
\StringTok{  }\KeywordTok{html_nodes}\NormalTok{(}\StringTok{"table:nth-child(14) td:nth-child(4)"}\NormalTok{) }\OperatorTok
\StringTok{  }\KeywordTok{html_text}\NormalTok{()}
\NormalTok{Lakes.Impaired.percent <-}\StringTok{ }\NormalTok{webpage }\OperatorTok
\StringTok{  }\KeywordTok{html_nodes}\NormalTok{(}\StringTok{"table:nth-child(14) td:nth-child(5)"}\NormalTok{) }\OperatorTok
\StringTok{  }\KeywordTok{html_text}\NormalTok{()}
\NormalTok{Lakes.Impaired.percent.TMDL <-}\StringTok{ }\NormalTok{webpage }\OperatorTok
\StringTok{  }\KeywordTok{html_nodes}\NormalTok{(}\StringTok{"table:nth-child(14) td:nth-child(6)"}\NormalTok{) }\OperatorTok
\StringTok{  }\KeywordTok{html_text}\NormalTok{()}

\NormalTok{Lakes <-}\StringTok{ }\KeywordTok{data.frame}\NormalTok{(State, Lakes.Assessed.acres, Lakes.Assessed.percent, }
\NormalTok{                    Lakes.Impaired.acres, Lakes.Impaired.percent, }
\NormalTok{                    Lakes.Impaired.percent.TMDL)}
\end{Highlighting}
\end{Shaded}

\begin{enumerate}
\def\labelenumi{\arabic{enumi}.}
\setcounter{enumi}{6}
\item
  Filter out the states with no data.
\item
  Use \texttt{str\_replace} to remove non-numeric characters from the
  numeric columns.
\item
  Set the numeric columns to a numeric class and verify this using
  \texttt{str}.
\end{enumerate}

\begin{Shaded}
\begin{Highlighting}[]
\CommentTok{# 7}
\NormalTok{Lakes <-}\StringTok{ }\NormalTok{Lakes }\OperatorTok
\StringTok{  }\KeywordTok{filter}\NormalTok{(State }\OperatorTok{!=}\StringTok{ "Hawaii"} \OperatorTok{&}\StringTok{ }\NormalTok{State }\OperatorTok{!=}\StringTok{ "Pennsylvania"}\NormalTok{)}


\CommentTok{# 8}
\NormalTok{Lakes}\OperatorTok{$}\NormalTok{Lakes.Assessed.acres<-}\StringTok{ }\KeywordTok{str_replace}\NormalTok{(Lakes}\OperatorTok{$}\NormalTok{Lakes.Assessed.acres, }
                                          \DataTypeTok{pattern =} \StringTok{"([,])"}\NormalTok{, }\DataTypeTok{replacement =} \StringTok{""}\NormalTok{)}
\NormalTok{Lakes}\OperatorTok{$}\NormalTok{Lakes.Assessed.acres <-}\StringTok{ }\KeywordTok{str_replace}\NormalTok{(Lakes}\OperatorTok{$}\NormalTok{Lakes.Assessed.acres, }
                                          \DataTypeTok{pattern =} \StringTok{"([.])"}\NormalTok{, }\DataTypeTok{replacement =} \StringTok{""}\NormalTok{)}
\NormalTok{Lakes}\OperatorTok{$}\NormalTok{Lakes.Assessed.percent <-}\StringTok{ }\KeywordTok{str_replace}\NormalTok{(Lakes}\OperatorTok{$}\NormalTok{Lakes.Assessed.percent,}
                                              \DataTypeTok{pattern =} \StringTok{"([%])"}\NormalTok{, }\DataTypeTok{replacement =} \StringTok{""}\NormalTok{)}
\NormalTok{Lakes}\OperatorTok{$}\NormalTok{Lakes.Assessed.percent <-}\StringTok{ }\KeywordTok{str_replace}\NormalTok{(Lakes}\OperatorTok{$}\NormalTok{Lakes.Assessed.percent, }
                                              \DataTypeTok{pattern =} \StringTok{"([*])"}\NormalTok{, }\DataTypeTok{replacement =} \StringTok{""}\NormalTok{)}
\NormalTok{Lakes}\OperatorTok{$}\NormalTok{Lakes.Impaired.acres <-}\StringTok{ }\KeywordTok{str_replace}\NormalTok{(Lakes}\OperatorTok{$}\NormalTok{Lakes.Impaired.acres,}
                                          \DataTypeTok{pattern =} \StringTok{"([,])"}\NormalTok{, }\DataTypeTok{replacement =} \StringTok{""}\NormalTok{)}
\NormalTok{Lakes}\OperatorTok{$}\NormalTok{Lakes.Impaired.percent <-}\StringTok{ }\KeywordTok{str_replace}\NormalTok{(Lakes}\OperatorTok{$}\NormalTok{Lakes.Impaired.percent,}
                                              \DataTypeTok{pattern =} \StringTok{"([%])"}\NormalTok{, }\DataTypeTok{replacement =} \StringTok{""}\NormalTok{)}
\NormalTok{Lakes}\OperatorTok{$}\NormalTok{Lakes.Impaired.percent.TMDL <-}\StringTok{ }\KeywordTok{str_replace}\NormalTok{(Lakes}\OperatorTok{$}\NormalTok{Lakes.Impaired.percent.TMDL,}
                                                   \DataTypeTok{pattern =} \StringTok{"([%])"}\NormalTok{, }\DataTypeTok{replacement =} \StringTok{""}\NormalTok{)}
\NormalTok{Lakes}\OperatorTok{$}\NormalTok{Lakes.Impaired.percent.TMDL <-}\StringTok{ }\KeywordTok{str_replace}\NormalTok{(Lakes}\OperatorTok{$}\NormalTok{Lakes.Impaired.percent.TMDL,}
                                                   \DataTypeTok{pattern =} \StringTok{"([±])"}\NormalTok{, }\DataTypeTok{replacement =} \StringTok{""}\NormalTok{)}

\CommentTok{# 9}
\KeywordTok{str}\NormalTok{(Lakes)}
\end{Highlighting}
\end{Shaded}

\begin{verbatim}
## 'data.frame':    48 obs. of  6 variables:
##  $ State                      : Factor w/ 50 levels "Alabama","Alaska",..: 1 2 3 4 5 6 7 8 9 10 ...
##  $ Lakes.Assessed.acres       : chr  "430976" "5981" "114976" "64778" ...
##  $ Lakes.Assessed.percent     : chr  "88" "0" "34" "13" ...
##  $ Lakes.Impaired.acres       : chr  "81740" "1137" "4895" "6513" ...
##  $ Lakes.Impaired.percent     : chr  "19" "19" "4" "10" ...
##  $ Lakes.Impaired.percent.TMDL: chr  "53" "73" "9" "71" ...
\end{verbatim}

\begin{Shaded}
\begin{Highlighting}[]
\NormalTok{Lakes}\OperatorTok{$}\NormalTok{Lakes.Assessed.acres <-}\StringTok{ }\KeywordTok{as.numeric}\NormalTok{(Lakes}\OperatorTok{$}\NormalTok{Lakes.Assessed.acres)}
\end{Highlighting}
\end{Shaded}

\begin{verbatim}
## Warning: NAs introduced by coercion
\end{verbatim}

\begin{Shaded}
\begin{Highlighting}[]
\NormalTok{Lakes}\OperatorTok{$}\NormalTok{Lakes.Assessed.percent <-}\StringTok{ }\KeywordTok{as.numeric}\NormalTok{(Lakes}\OperatorTok{$}\NormalTok{Lakes.Assessed.percent)}
\NormalTok{Lakes}\OperatorTok{$}\NormalTok{Lakes.Impaired.acres <-}\StringTok{ }\KeywordTok{as.numeric}\NormalTok{(Lakes}\OperatorTok{$}\NormalTok{Lakes.Impaired.acres)}
\NormalTok{Lakes}\OperatorTok{$}\NormalTok{Lakes.Impaired.percent <-}\StringTok{ }\KeywordTok{as.numeric}\NormalTok{(Lakes}\OperatorTok{$}\NormalTok{Lakes.Impaired.percent)}
\NormalTok{Lakes}\OperatorTok{$}\NormalTok{Lakes.Impaired.percent.TMDL <-}\StringTok{ }\KeywordTok{as.numeric}\NormalTok{(Lakes}\OperatorTok{$}\NormalTok{Lakes.Impaired.percent.TMDL)}

\KeywordTok{str}\NormalTok{(Lakes)}
\end{Highlighting}
\end{Shaded}

\begin{verbatim}
## 'data.frame':    48 obs. of  6 variables:
##  $ State                      : Factor w/ 50 levels "Alabama","Alaska",..: 1 2 3 4 5 6 7 8 9 10 ...
##  $ Lakes.Assessed.acres       : num  430976 5981 114976 64778 NA ...
##  $ Lakes.Assessed.percent     : num  88 0 34 13 50 95 47 100 54 82 ...
##  $ Lakes.Impaired.acres       : num  81740 1137 4895 6513 473954 ...
##  $ Lakes.Impaired.percent     : num  19 19 4 10 45 7 12 88 82 2 ...
##  $ Lakes.Impaired.percent.TMDL: num  53 73 9 71 NA 0 7 69 NA 20 ...
\end{verbatim}

\begin{enumerate}
\def\labelenumi{\arabic{enumi}.}
\setcounter{enumi}{9}
\tightlist
\item
  Join the two data frames with a \texttt{full\_join}.
\end{enumerate}

\begin{Shaded}
\begin{Highlighting}[]
\NormalTok{Rivers_Lakes <-}\StringTok{ }\KeywordTok{full_join}\NormalTok{(Rivers, Lakes)}
\end{Highlighting}
\end{Shaded}

\begin{verbatim}
## Joining, by = "State"
\end{verbatim}

\begin{enumerate}
\def\labelenumi{\arabic{enumi}.}
\setcounter{enumi}{10}
\tightlist
\item
  Create one graph that compares the data for lakes and/or rivers. This
  option is flexible; choose a relationship (or relationships) that seem
  interesting to you, and think about the implications of your findings.
  This graph should be edited so it follows best data visualization
  practices.
\end{enumerate}

(You may choose to run a statistical test or add a line of best fit;
this is optional but may aid in your interpretations)

\begin{Shaded}
\begin{Highlighting}[]
\CommentTok{#Creating a gathered data frame to make a boxplot}
\NormalTok{Rivers_Lakes_boxplot <-}\StringTok{ }\NormalTok{Rivers_Lakes }\OperatorTok\StringTok{ }
\StringTok{  }\KeywordTok{select}\NormalTok{(State, Rivers.Impaired.percent, Lakes.Impaired.percent) }\OperatorTok\StringTok{ }
\StringTok{  }\KeywordTok{rename}\NormalTok{(}\DataTypeTok{Rivers =}\NormalTok{ Rivers.Impaired.percent, }\DataTypeTok{Lakes =}\NormalTok{ Lakes.Impaired.percent) }\OperatorTok
\StringTok{  }\KeywordTok{gather}\NormalTok{(Feature, Impaired.percent, }\OperatorTok{-}\NormalTok{State)}

\CommentTok{#Make plot}
\NormalTok{rivers_lakes_box <-}\StringTok{ }\KeywordTok{ggplot}\NormalTok{(Rivers_Lakes_boxplot, }
                   \KeywordTok{aes}\NormalTok{(}\DataTypeTok{y =}\NormalTok{ Impaired.percent, }\DataTypeTok{x =} \KeywordTok{as.factor}\NormalTok{(Feature))) }\OperatorTok{+}
\StringTok{  }\KeywordTok{geom_boxplot}\NormalTok{() }\OperatorTok{+}\StringTok{ }
\StringTok{  }\KeywordTok{labs}\NormalTok{(}\DataTypeTok{y =} \StringTok{"% Impaired, by US State"}\NormalTok{, }\DataTypeTok{x =} \StringTok{"Feature"}\NormalTok{) }\OperatorTok{+}
\StringTok{  }\KeywordTok{ylim}\NormalTok{(}\DecValTok{0}\NormalTok{, }\DecValTok{100}\NormalTok{)}

\KeywordTok{print}\NormalTok{(rivers_lakes_box)}
\end{Highlighting}
\end{Shaded}

\begin{verbatim}
## Warning: Removed 2 rows containing non-finite values (stat_boxplot).
\end{verbatim}

\includegraphics{A10_DataScraping_files/figure-latex/unnamed-chunk-8-1.pdf}

\begin{Shaded}
\begin{Highlighting}[]
\CommentTok{#Calculations to aid in interpretation}
\KeywordTok{summary}\NormalTok{(Rivers_Lakes}\OperatorTok{$}\NormalTok{Rivers.Impaired.percent)}
\end{Highlighting}
\end{Shaded}

\begin{verbatim}
##    Min. 1st Qu.  Median    Mean 3rd Qu.    Max. 
##    0.00    3.25    9.50   17.42   21.50   88.00
\end{verbatim}

\begin{Shaded}
\begin{Highlighting}[]
\KeywordTok{summary}\NormalTok{(Rivers_Lakes}\OperatorTok{$}\NormalTok{Lakes.Impaired.percent, }\DataTypeTok{na.rm=}\OtherTok{TRUE}\NormalTok{)}
\end{Highlighting}
\end{Shaded}

\begin{verbatim}
##    Min. 1st Qu.  Median    Mean 3rd Qu.    Max.    NA's 
##    0.00    7.00   18.50   28.19   41.25   91.00       2
\end{verbatim}

\begin{Shaded}
\begin{Highlighting}[]
\KeywordTok{var.test}\NormalTok{(Rivers_Lakes_boxplot}\OperatorTok{$}\NormalTok{Impaired.percent }\OperatorTok{~}\StringTok{ }\KeywordTok{as.factor}\NormalTok{(Rivers_Lakes_boxplot }\OperatorTok{$}\NormalTok{Feature))}
\end{Highlighting}
\end{Shaded}

\begin{verbatim}
## 
##  F test to compare two variances
## 
## data:  Rivers_Lakes_boxplot$Impaired.percent by as.factor(Rivers_Lakes_boxplot$Feature)
## F = 2.0013, num df = 47, denom df = 49, p-value = 0.01756
## alternative hypothesis: true ratio of variances is not equal to 1
## 95 percent confidence interval:
##  1.130676 3.554481
## sample estimates:
## ratio of variances 
##            2.00128
\end{verbatim}

\begin{Shaded}
\begin{Highlighting}[]
\KeywordTok{t.test}\NormalTok{(Rivers_Lakes_boxplot}\OperatorTok{$}\NormalTok{Impaired.percent }\OperatorTok{~}\StringTok{ }
\StringTok{         }\KeywordTok{as.factor}\NormalTok{(Rivers_Lakes_boxplot }\OperatorTok{$}\NormalTok{Feature), }\DataTypeTok{var.equal =} \OtherTok{FALSE}\NormalTok{)}
\end{Highlighting}
\end{Shaded}

\begin{verbatim}
## 
##  Welch Two Sample t-test
## 
## data:  Rivers_Lakes_boxplot$Impaired.percent by as.factor(Rivers_Lakes_boxplot$Feature)
## t = 2.182, df = 84.3, p-value = 0.03189
## alternative hypothesis: true difference in means is not equal to 0
## 95 percent confidence interval:
##   0.9550311 20.5799689
## sample estimates:
##  mean in group Lakes mean in group Rivers 
##              28.1875              17.4200
\end{verbatim}

\begin{enumerate}
\def\labelenumi{\arabic{enumi}.}
\setcounter{enumi}{11}
\tightlist
\item
  Summarize the findings that accompany your graph. You may choose to
  suggest further research or data collection to help explain the
  results.
\end{enumerate}

\begin{quote}
A boxplot comparing percentages of lakes and rivers impaired in states
across the US shows that there is a greater median percentage of
impaired lakes (18.5\%) than impaired rivers (9.5\%). The distributions
of impaired percentages for both features are positively skewed, but the
rivers data is less spread out, with a smaller IQR than the lakes data.
These features of the boxplot suggest that the state percentages of
impaired lakes range more widely than those of impaired rivers and that,
overall, there are greater percentages of impaired lakes than rivers. A
t-test confirms that the average percentage of impaired lakes is
significantly greater than the average percentage of impaired rivers in
US states (t = 2.182, df = 84.3, p \textless{} 0.05), indicating that US
lakes may often need to be prioritized in nutrient-related preservation
and restoration efforts.
\end{quote}


\end{document}
