\documentclass[]{article}
\usepackage{lmodern}
\usepackage{amssymb,amsmath}
\usepackage{ifxetex,ifluatex}
\usepackage{fixltx2e} % provides \textsubscript
\ifnum 0\ifxetex 1\fi\ifluatex 1\fi=0 % if pdftex
  \usepackage[T1]{fontenc}
  \usepackage[utf8]{inputenc}
\else % if luatex or xelatex
  \ifxetex
    \usepackage{mathspec}
  \else
    \usepackage{fontspec}
  \fi
  \defaultfontfeatures{Ligatures=TeX,Scale=MatchLowercase}
\fi
% use upquote if available, for straight quotes in verbatim environments
\IfFileExists{upquote.sty}{\usepackage{upquote}}{}
% use microtype if available
\IfFileExists{microtype.sty}{%
\usepackage{microtype}
\UseMicrotypeSet[protrusion]{basicmath} % disable protrusion for tt fonts
}{}
\usepackage[margin=2.54cm]{geometry}
\usepackage{hyperref}
\hypersetup{unicode=true,
            pdftitle={Assignment 3: Data Exploration},
            pdfauthor={Claire Mullaney},
            pdfborder={0 0 0},
            breaklinks=true}
\urlstyle{same}  % don't use monospace font for urls
\usepackage{color}
\usepackage{fancyvrb}
\newcommand{\VerbBar}{|}
\newcommand{\VERB}{\Verb[commandchars=\\\{\}]}
\DefineVerbatimEnvironment{Highlighting}{Verbatim}{commandchars=\\\{\}}
% Add ',fontsize=\small' for more characters per line
\usepackage{framed}
\definecolor{shadecolor}{RGB}{248,248,248}
\newenvironment{Shaded}{\begin{snugshade}}{\end{snugshade}}
\newcommand{\AlertTok}[1]{\textcolor[rgb]{0.94,0.16,0.16}{#1}}
\newcommand{\AnnotationTok}[1]{\textcolor[rgb]{0.56,0.35,0.01}{\textbf{\textit{#1}}}}
\newcommand{\AttributeTok}[1]{\textcolor[rgb]{0.77,0.63,0.00}{#1}}
\newcommand{\BaseNTok}[1]{\textcolor[rgb]{0.00,0.00,0.81}{#1}}
\newcommand{\BuiltInTok}[1]{#1}
\newcommand{\CharTok}[1]{\textcolor[rgb]{0.31,0.60,0.02}{#1}}
\newcommand{\CommentTok}[1]{\textcolor[rgb]{0.56,0.35,0.01}{\textit{#1}}}
\newcommand{\CommentVarTok}[1]{\textcolor[rgb]{0.56,0.35,0.01}{\textbf{\textit{#1}}}}
\newcommand{\ConstantTok}[1]{\textcolor[rgb]{0.00,0.00,0.00}{#1}}
\newcommand{\ControlFlowTok}[1]{\textcolor[rgb]{0.13,0.29,0.53}{\textbf{#1}}}
\newcommand{\DataTypeTok}[1]{\textcolor[rgb]{0.13,0.29,0.53}{#1}}
\newcommand{\DecValTok}[1]{\textcolor[rgb]{0.00,0.00,0.81}{#1}}
\newcommand{\DocumentationTok}[1]{\textcolor[rgb]{0.56,0.35,0.01}{\textbf{\textit{#1}}}}
\newcommand{\ErrorTok}[1]{\textcolor[rgb]{0.64,0.00,0.00}{\textbf{#1}}}
\newcommand{\ExtensionTok}[1]{#1}
\newcommand{\FloatTok}[1]{\textcolor[rgb]{0.00,0.00,0.81}{#1}}
\newcommand{\FunctionTok}[1]{\textcolor[rgb]{0.00,0.00,0.00}{#1}}
\newcommand{\ImportTok}[1]{#1}
\newcommand{\InformationTok}[1]{\textcolor[rgb]{0.56,0.35,0.01}{\textbf{\textit{#1}}}}
\newcommand{\KeywordTok}[1]{\textcolor[rgb]{0.13,0.29,0.53}{\textbf{#1}}}
\newcommand{\NormalTok}[1]{#1}
\newcommand{\OperatorTok}[1]{\textcolor[rgb]{0.81,0.36,0.00}{\textbf{#1}}}
\newcommand{\OtherTok}[1]{\textcolor[rgb]{0.56,0.35,0.01}{#1}}
\newcommand{\PreprocessorTok}[1]{\textcolor[rgb]{0.56,0.35,0.01}{\textit{#1}}}
\newcommand{\RegionMarkerTok}[1]{#1}
\newcommand{\SpecialCharTok}[1]{\textcolor[rgb]{0.00,0.00,0.00}{#1}}
\newcommand{\SpecialStringTok}[1]{\textcolor[rgb]{0.31,0.60,0.02}{#1}}
\newcommand{\StringTok}[1]{\textcolor[rgb]{0.31,0.60,0.02}{#1}}
\newcommand{\VariableTok}[1]{\textcolor[rgb]{0.00,0.00,0.00}{#1}}
\newcommand{\VerbatimStringTok}[1]{\textcolor[rgb]{0.31,0.60,0.02}{#1}}
\newcommand{\WarningTok}[1]{\textcolor[rgb]{0.56,0.35,0.01}{\textbf{\textit{#1}}}}
\usepackage{graphicx,grffile}
\makeatletter
\def\maxwidth{\ifdim\Gin@nat@width>\linewidth\linewidth\else\Gin@nat@width\fi}
\def\maxheight{\ifdim\Gin@nat@height>\textheight\textheight\else\Gin@nat@height\fi}
\makeatother
% Scale images if necessary, so that they will not overflow the page
% margins by default, and it is still possible to overwrite the defaults
% using explicit options in \includegraphics[width, height, ...]{}
\setkeys{Gin}{width=\maxwidth,height=\maxheight,keepaspectratio}
\IfFileExists{parskip.sty}{%
\usepackage{parskip}
}{% else
\setlength{\parindent}{0pt}
\setlength{\parskip}{6pt plus 2pt minus 1pt}
}
\setlength{\emergencystretch}{3em}  % prevent overfull lines
\providecommand{\tightlist}{%
  \setlength{\itemsep}{0pt}\setlength{\parskip}{0pt}}
\setcounter{secnumdepth}{0}
% Redefines (sub)paragraphs to behave more like sections
\ifx\paragraph\undefined\else
\let\oldparagraph\paragraph
\renewcommand{\paragraph}[1]{\oldparagraph{#1}\mbox{}}
\fi
\ifx\subparagraph\undefined\else
\let\oldsubparagraph\subparagraph
\renewcommand{\subparagraph}[1]{\oldsubparagraph{#1}\mbox{}}
\fi

%%% Use protect on footnotes to avoid problems with footnotes in titles
\let\rmarkdownfootnote\footnote%
\def\footnote{\protect\rmarkdownfootnote}

%%% Change title format to be more compact
\usepackage{titling}

% Create subtitle command for use in maketitle
\providecommand{\subtitle}[1]{
  \posttitle{
    \begin{center}\large#1\end{center}
    }
}

\setlength{\droptitle}{-2em}

  \title{Assignment 3: Data Exploration}
    \pretitle{\vspace{\droptitle}\centering\huge}
  \posttitle{\par}
    \author{Claire Mullaney}
    \preauthor{\centering\large\emph}
  \postauthor{\par}
    \date{}
    \predate{}\postdate{}
  

\begin{document}
\maketitle

\hypertarget{overview}{%
\subsection{OVERVIEW}\label{overview}}

This exercise accompanies the lessons in Environmental Data Analytics on
Data Exploration.

\hypertarget{directions}{%
\subsection{Directions}\label{directions}}

\begin{enumerate}
\def\labelenumi{\arabic{enumi}.}
\tightlist
\item
  Change ``Student Name'' on line 3 (above) with your name.
\item
  Work through the steps, \textbf{creating code and output} that fulfill
  each instruction.
\item
  Be sure to \textbf{answer the questions} in this assignment document.
\item
  When you have completed the assignment, \textbf{Knit} the text and
  code into a single PDF file.
\item
  After Knitting, submit the completed exercise (PDF file) to the
  dropbox in Sakai. Add your last name into the file name (e.g.,
  ``Salk\_A03\_DataExploration.Rmd'') prior to submission.
\end{enumerate}

The completed exercise is due on Tuesday, January 28 at 1:00 pm.

\hypertarget{set-up-your-r-session}{%
\subsection{Set up your R session}\label{set-up-your-r-session}}

\begin{enumerate}
\def\labelenumi{\arabic{enumi}.}
\tightlist
\item
  Check your working directory, load necessary packages (tidyverse), and
  upload two datasets: the ECOTOX neonicotinoid dataset
  (ECOTOX\_Neonicotinoids\_Insects\_raw.csv) and the Niwot Ridge NEON
  dataset for litter and woody debris
  (NEON\_NIWO\_Litter\_massdata\_2018-08\_raw.csv). Name these datasets
  ``Neonics'' and ``Litter'', respectively.
\end{enumerate}

\begin{Shaded}
\begin{Highlighting}[]
\CommentTok{#Check working directory}
\KeywordTok{getwd}\NormalTok{()}
\end{Highlighting}
\end{Shaded}

\begin{verbatim}
## [1] "/Users/clairemullaney/Desktop/ENV 872/Environmental_Data_Analytics_2020"
\end{verbatim}

\begin{Shaded}
\begin{Highlighting}[]
\CommentTok{#Load necessary packages}
\KeywordTok{library}\NormalTok{(tidyverse)}

\CommentTok{#Load datasets}
\NormalTok{Neonics <-}\StringTok{ }\KeywordTok{read.csv}\NormalTok{(}\StringTok{"./Data/Raw/ECOTOX_Neonicotinoids_Insects_raw.csv"}\NormalTok{)}

\NormalTok{Litter <-}\StringTok{ }\KeywordTok{read.csv}\NormalTok{(}\StringTok{"./Data/Raw/NEON_NIWO_Litter_massdata_2018-08_raw.csv"}\NormalTok{)}
\end{Highlighting}
\end{Shaded}

\hypertarget{learn-about-your-system}{%
\subsection{Learn about your system}\label{learn-about-your-system}}

\begin{enumerate}
\def\labelenumi{\arabic{enumi}.}
\setcounter{enumi}{1}
\tightlist
\item
  The neonicotinoid dataset was collected from the Environmental
  Protection Agency's ECOTOX Knowledgebase, a database for ecotoxicology
  research. Neonicotinoids are a class of insecticides used widely in
  agriculture. The dataset that has been pulled includes all studies
  published on insects. Why might we be interested in the ecotoxicologoy
  of neonicotinoids on insects? Feel free to do a brief internet search
  if you feel you need more background information.
\end{enumerate}

\begin{quote}
Answer: Some insects play important ecosystem roles and perform tasks
cruicial to human health and survival -- for example, they pollinate
flowers and crops. If neonicotinoids are possibly harming beneficial
insects and preventing them from effectively completing these tasks, the
impact of neonicotinoids on insect health would need to be studied. From
a quick search, it appears that there is evidence that neonicotinoids
are contributing to declines in bee populations, which would certainly
cause scientists to devote resources to this area of study.
\end{quote}

\begin{enumerate}
\def\labelenumi{\arabic{enumi}.}
\setcounter{enumi}{2}
\tightlist
\item
  The Niwot Ridge litter and woody debris dataset was collected from the
  National Ecological Observatory Network, which collectively includes
  81 aquatic and terrestrial sites across 20 ecoclimatic domains. 32 of
  these sites sample forest litter and woody debris, and we will focus
  on the Niwot Ridge long-term ecological research (LTER) station in
  Colorado. Why might we be interested in studying litter and woody
  debris that falls to the ground in forests? Feel free to do a brief
  internet search if you feel you need more background information.
\end{enumerate}

\begin{quote}
Answer: Litter and woody debris that fall to the ground in forests could
be used to assess different element and chemical concentrations in the
foliage. These concentrations could possibly be indicators of overall
forest health, and anomolies or trends could be indicative of an event
that caused the litter to contain more of a specific compound or element
(e.g., deposition of mercury from anthropogenic emissions). These debris
could also be used to study ecological productivity, nutrient and carbon
cycling, and soil fertility.
\end{quote}

\begin{enumerate}
\def\labelenumi{\arabic{enumi}.}
\setcounter{enumi}{3}
\tightlist
\item
  How is litter and woody debris sampled as part of the NEON network?
  Read the NEON\_Litterfall\_UserGuide.pdf document to learn more. List
  three pieces of salient information about the sampling methods here:
\end{enumerate}

\begin{quote}
Answer:
\end{quote}

\begin{quote}
\begin{enumerate}
\def\labelenumi{\arabic{enumi}.}
\tightlist
\item
  Litter and woody debris were collected using both elevated and ground
  traps. Elevated traps primarily catch litter, and each is a 0.5 m\^{}2
  square with a mesh basket elevated about 80 cm above the ground.
  Ground traps primarily catch woody debris, and each is 3 m x 0.5 m.
\end{enumerate}
\end{quote}

\begin{quote}
\begin{enumerate}
\def\labelenumi{\arabic{enumi}.}
\setcounter{enumi}{1}
\tightlist
\item
  Sampling occurs only in plots near towers. If the area near the tower
  is forested, litter sampling takes place in 20 40m x 40m plots. If the
  sites near the tower have low-statured vegetation, litter sampling
  occurs in 4 40m x 40m tower plots as well as 26 20m x 20m plots. One
  elevated trap and one ground trap are deployed for each 400 m\^{}2
  plot area.
\end{enumerate}
\end{quote}

\begin{quote}
\begin{enumerate}
\def\labelenumi{\arabic{enumi}.}
\setcounter{enumi}{2}
\tightlist
\item
  Ground traps are sampled once per year, while sampling of elevated
  traps depends on the vegetation that is present (deciduous forest
  sites are sampled once every 2 weeks during senescence and evergreen
  sites are sampled once every 1-2 months; some deciduous sites or sites
  that are hard to access may not be sampled for up to 6 months over the
  winter).
\end{enumerate}
\end{quote}

\hypertarget{obtain-basic-summaries-of-your-data-neonics}{%
\subsection{Obtain basic summaries of your data
(Neonics)}\label{obtain-basic-summaries-of-your-data-neonics}}

\begin{enumerate}
\def\labelenumi{\arabic{enumi}.}
\setcounter{enumi}{4}
\tightlist
\item
  What are the dimensions of the dataset?
\end{enumerate}

\begin{Shaded}
\begin{Highlighting}[]
\CommentTok{#Dimensions of the Neonics dataset}
\KeywordTok{dim}\NormalTok{(Neonics)}
\end{Highlighting}
\end{Shaded}

\begin{verbatim}
## [1] 4623   30
\end{verbatim}

\begin{enumerate}
\def\labelenumi{\arabic{enumi}.}
\setcounter{enumi}{5}
\tightlist
\item
  Using the \texttt{summary} function, determine the most common effects
  that are studied. Why might these effects specifically be of interest?
\end{enumerate}

\begin{Shaded}
\begin{Highlighting}[]
\CommentTok{#Determining the most commonly studied effects}
\KeywordTok{summary}\NormalTok{(Neonics}\OperatorTok{$}\NormalTok{Effect)}
\end{Highlighting}
\end{Shaded}

\begin{verbatim}
##     Accumulation        Avoidance         Behavior     Biochemistry 
##               12              102              360               11 
##          Cell(s)      Development        Enzyme(s) Feeding behavior 
##                9              136               62              255 
##         Genetics           Growth        Histology       Hormone(s) 
##               82               38                5                1 
##    Immunological     Intoxication       Morphology        Mortality 
##               16               12               22             1493 
##       Physiology       Population     Reproduction 
##                7             1803              197
\end{verbatim}

\begin{quote}
Answer: The most common effects that are studied are population and
mortality (the only two effects that are examined by over 1,000
publications). While it would be illuminating to study other effects as
well, the health and abundance of insect populations are of ultimate
importance; if many members of a species of insect are dying (mortality)
or large scale population shifts/changes are occurring, these impacts
would most definitively signal the need for action, as they indicate
that these insects may not be able to continue providing their current
services and benefits.
\end{quote}

\begin{enumerate}
\def\labelenumi{\arabic{enumi}.}
\setcounter{enumi}{6}
\tightlist
\item
  Using the \texttt{summary} function, determine the six most commonly
  studied species in the dataset (common name). What do these species
  have in common, and why might they be of interest over other insects?
  Feel free to do a brief internet search for more information if
  needed.
\end{enumerate}

\begin{Shaded}
\begin{Highlighting}[]
\KeywordTok{summary}\NormalTok{(Neonics}\OperatorTok{$}\NormalTok{Species.Common.Name)}
\end{Highlighting}
\end{Shaded}

\begin{verbatim}
##                          Honey Bee                     Parasitic Wasp 
##                                667                                285 
##              Buff Tailed Bumblebee                Carniolan Honey Bee 
##                                183                                152 
##                         Bumble Bee                   Italian Honeybee 
##                                140                                113 
##                    Japanese Beetle                  Asian Lady Beetle 
##                                 94                                 76 
##                     Euonymus Scale                           Wireworm 
##                                 75                                 69 
##                  European Dark Bee                  Minute Pirate Bug 
##                                 66                                 62 
##               Asian Citrus Psyllid                      Parastic Wasp 
##                                 60                                 58 
##             Colorado Potato Beetle                    Parasitoid Wasp 
##                                 57                                 51 
##                Erythrina Gall Wasp                       Beetle Order 
##                                 49                                 47 
##        Snout Beetle Family, Weevil           Sevenspotted Lady Beetle 
##                                 47                                 46 
##                     True Bug Order              Buff-tailed Bumblebee 
##                                 45                                 39 
##                       Aphid Family                     Cabbage Looper 
##                                 38                                 38 
##               Sweetpotato Whitefly                      Braconid Wasp 
##                                 37                                 33 
##                       Cotton Aphid                     Predatory Mite 
##                                 33                                 33 
##             Ladybird Beetle Family                         Parasitoid 
##                                 30                                 30 
##                      Scarab Beetle                      Spring Tiphia 
##                                 29                                 29 
##                        Thrip Order               Ground Beetle Family 
##                                 29                                 27 
##                 Rove Beetle Family                      Tobacco Aphid 
##                                 27                                 27 
##                       Chalcid Wasp             Convergent Lady Beetle 
##                                 25                                 25 
##                      Stingless Bee                  Spider/Mite Class 
##                                 25                                 24 
##                Tobacco Flea Beetle                   Citrus Leafminer 
##                                 24                                 23 
##                    Ladybird Beetle                          Mason Bee 
##                                 23                                 22 
##                           Mosquito                      Argentine Ant 
##                                 22                                 21 
##                             Beetle         Flatheaded Appletree Borer 
##                                 21                                 20 
##               Horned Oak Gall Wasp                 Leaf Beetle Family 
##                                 20                                 20 
##                  Potato Leafhopper         Tooth-necked Fungus Beetle 
##                                 20                                 20 
##                       Codling Moth          Black-spotted Lady Beetle 
##                                 19                                 18 
##                       Calico Scale                Fairyfly Parasitoid 
##                                 18                                 18 
##                        Lady Beetle             Minute Parasitic Wasps 
##                                 18                                 18 
##                          Mirid Bug                   Mulberry Pyralid 
##                                 18                                 18 
##                           Silkworm                     Vedalia Beetle 
##                                 18                                 18 
##              Araneoid Spider Order                          Bee Order 
##                                 17                                 17 
##                     Egg Parasitoid                       Insect Class 
##                                 17                                 17 
##           Moth And Butterfly Order       Oystershell Scale Parasitoid 
##                                 17                                 17 
## Hemlock Woolly Adelgid Lady Beetle              Hemlock Wooly Adelgid 
##                                 16                                 16 
##                               Mite                        Onion Thrip 
##                                 16                                 16 
##              Western Flower Thrips                       Corn Earworm 
##                                 15                                 14 
##                  Green Peach Aphid                          House Fly 
##                                 14                                 14 
##                          Ox Beetle                 Red Scale Parasite 
##                                 14                                 14 
##                 Spined Soldier Bug              Armoured Scale Family 
##                                 14                                 13 
##                   Diamondback Moth                      Eulophid Wasp 
##                                 13                                 13 
##                  Monarch Butterfly                      Predatory Bug 
##                                 13                                 13 
##              Yellow Fever Mosquito                Braconid Parasitoid 
##                                 13                                 12 
##                       Common Thrip       Eastern Subterranean Termite 
##                                 12                                 12 
##                             Jassid                         Mite Order 
##                                 12                                 12 
##                          Pea Aphid                   Pond Wolf Spider 
##                                 12                                 12 
##           Spotless Ladybird Beetle             Glasshouse Potato Wasp 
##                                 11                                 10 
##                           Lacewing            Southern House Mosquito 
##                                 10                                 10 
##            Two Spotted Lady Beetle                         Ant Family 
##                                 10                                  9 
##                       Apple Maggot                            (Other) 
##                                  9                                670
\end{verbatim}

\begin{quote}
Answer: The six most commonly studied species in the data set are: Honey
Bee, Parasitic Wasp, Buff Tailed Bumblebee, Carniolan Honey Bee, Bumble
Bee, Italian Honeybee. These insects are all beneficial in certain areas
of the world (although a couple of them, the honey bee and the buff
tailed bumblebee, have become invasive); the bees are pollinators, while
the parasitic wasp helps control populations of insects that harm
gardens and crops.
\end{quote}

\begin{enumerate}
\def\labelenumi{\arabic{enumi}.}
\setcounter{enumi}{7}
\tightlist
\item
  Concentrations are always a numeric value. What is the class of
  Conc.1..Author. in the dataset, and why is it not numeric?
\end{enumerate}

\begin{Shaded}
\begin{Highlighting}[]
\CommentTok{#Finding the class of Conc.1..Author.}
\KeywordTok{class}\NormalTok{(Neonics}\OperatorTok{$}\NormalTok{Conc.}\DecValTok{1}\NormalTok{..Author.)}
\end{Highlighting}
\end{Shaded}

\begin{verbatim}
## [1] "factor"
\end{verbatim}

\begin{quote}
Answer: The class of Conc.1..Author. in the dataset is a factor. Rather
than viewing these concentrations as numerical values on which
mathematical operations can be performed, this variable is meant to show
a limited number of concentration levels that can be seen as categories
or groups. In this specific dataset, it is more helpful to see how many
studies use specific concentration levels (perhaps for the planning of
future studies) as opposed to mathematically manipulating those
concentrations.
\end{quote}

\hypertarget{explore-your-data-graphically-neonics}{%
\subsection{Explore your data graphically
(Neonics)}\label{explore-your-data-graphically-neonics}}

\begin{enumerate}
\def\labelenumi{\arabic{enumi}.}
\setcounter{enumi}{8}
\tightlist
\item
  Using \texttt{geom\_freqpoly}, generate a plot of the number of
  studies conducted by publication year.
\end{enumerate}

\begin{Shaded}
\begin{Highlighting}[]
\KeywordTok{ggplot}\NormalTok{(Neonics) }\OperatorTok{+}\StringTok{ }
\StringTok{  }\KeywordTok{geom_freqpoly}\NormalTok{(}\KeywordTok{aes}\NormalTok{(}\DataTypeTok{x =}\NormalTok{ Publication.Year)) }\OperatorTok{+}\StringTok{ }
\StringTok{  }\KeywordTok{labs}\NormalTok{(}\DataTypeTok{x =} \StringTok{"Publication Year"}\NormalTok{, }\DataTypeTok{y =} \StringTok{"Number of Publications"}\NormalTok{)}
\end{Highlighting}
\end{Shaded}

\begin{verbatim}
## `stat_bin()` using `bins = 30`. Pick better value with `binwidth`.
\end{verbatim}

\includegraphics{A03_DataExploration_files/figure-latex/unnamed-chunk-6-1.pdf}

\begin{enumerate}
\def\labelenumi{\arabic{enumi}.}
\setcounter{enumi}{9}
\tightlist
\item
  Reproduce the same graph but now add a color aesthetic so that
  different Test.Location are displayed as different colors.
\end{enumerate}

\begin{Shaded}
\begin{Highlighting}[]
\KeywordTok{ggplot}\NormalTok{(Neonics) }\OperatorTok{+}\StringTok{ }
\StringTok{  }\KeywordTok{geom_freqpoly}\NormalTok{(}\KeywordTok{aes}\NormalTok{(}\DataTypeTok{x =}\NormalTok{ Publication.Year, }\DataTypeTok{color =}\NormalTok{ Test.Location)) }\OperatorTok{+}\StringTok{ }
\StringTok{  }\KeywordTok{labs}\NormalTok{(}\DataTypeTok{x =} \StringTok{"Publication Year"}\NormalTok{) }\OperatorTok{+}
\StringTok{  }\KeywordTok{theme}\NormalTok{(}\DataTypeTok{legend.position =} \StringTok{"top"}\NormalTok{) }\OperatorTok{+}
\StringTok{  }\KeywordTok{scale_color_discrete}\NormalTok{(}\DataTypeTok{name =} \StringTok{"Test Location"}\NormalTok{)}
\end{Highlighting}
\end{Shaded}

\begin{verbatim}
## `stat_bin()` using `bins = 30`. Pick better value with `binwidth`.
\end{verbatim}

\includegraphics{A03_DataExploration_files/figure-latex/unnamed-chunk-7-1.pdf}

Interpret this graph. What are the most common test locations, and do
they differ over time?

\begin{quote}
Answer: The most common test locations are the lab and natural field
environments. The number of experiments done in the lab matches the
overall frequency of publications fairly closely, increasing overall
from 1990 to about 2014 (with some decreases also present during this
time period), when it abruptly drops. The number of experiments done in
the natural field increases and decreases in cycles starting from 1990
to about 2007, spikes in about 2009, and drops off from there.
\end{quote}

\begin{enumerate}
\def\labelenumi{\arabic{enumi}.}
\setcounter{enumi}{10}
\tightlist
\item
  Create a bar graph of Endpoint counts. What are the two most common
  end points, and how are they defined? Consult the ECOTOX\_CodeAppendix
  for more information.
\end{enumerate}

\begin{Shaded}
\begin{Highlighting}[]
\KeywordTok{ggplot}\NormalTok{(Neonics, }\KeywordTok{aes}\NormalTok{(}\DataTypeTok{x =}\NormalTok{ Endpoint)) }\OperatorTok{+}\StringTok{ }
\StringTok{  }\KeywordTok{geom_bar}\NormalTok{()}
\end{Highlighting}
\end{Shaded}

\includegraphics{A03_DataExploration_files/figure-latex/unnamed-chunk-8-1.pdf}

\begin{quote}
Answer: The most common endpoint is NOEL. NOEL indicates that there is
no-observable-effect-level; the highest concentration of the
administered chemical produces effects that are not significantly
different from the responses of controls (according to the author's
reported statistical test). The second-most-common endpoint is LOEL.
LOEL represents the category lowest-observable-effect-level; the lowest
concentration of the administered chemical produces effects that are
significantly different from control responses.
\end{quote}

\hypertarget{explore-your-data-litter}{%
\subsection{Explore your data (Litter)}\label{explore-your-data-litter}}

\begin{enumerate}
\def\labelenumi{\arabic{enumi}.}
\setcounter{enumi}{11}
\tightlist
\item
  Determine the class of collectDate. Is it a date? If not, change to a
  date and confirm the new class of the variable. Using the
  \texttt{unique} function, determine which dates litter was sampled in
  August 2018.
\end{enumerate}

\begin{Shaded}
\begin{Highlighting}[]
\CommentTok{#Checking the class of the data}
\KeywordTok{class}\NormalTok{(Litter}\OperatorTok{$}\NormalTok{collectDate)}
\end{Highlighting}
\end{Shaded}

\begin{verbatim}
## [1] "factor"
\end{verbatim}

\begin{Shaded}
\begin{Highlighting}[]
\CommentTok{#CollectDate is a factor; changing it to a date:}
\NormalTok{Litter}\OperatorTok{$}\NormalTok{collectDate <-}\StringTok{ }\KeywordTok{as.Date}\NormalTok{(Litter}\OperatorTok{$}\NormalTok{collectDate, }\DataTypeTok{format =} \StringTok{"%Y-%m-%d"}\NormalTok{)}

\CommentTok{#Confirming the new class of the variable}
\KeywordTok{class}\NormalTok{(Litter}\OperatorTok{$}\NormalTok{collectDate)}
\end{Highlighting}
\end{Shaded}

\begin{verbatim}
## [1] "Date"
\end{verbatim}

\begin{Shaded}
\begin{Highlighting}[]
\CommentTok{#Dates litter was sampled in August of 2018:}
\KeywordTok{unique}\NormalTok{(Litter}\OperatorTok{$}\NormalTok{collectDate)}
\end{Highlighting}
\end{Shaded}

\begin{verbatim}
## [1] "2018-08-02" "2018-08-30"
\end{verbatim}

\begin{enumerate}
\def\labelenumi{\arabic{enumi}.}
\setcounter{enumi}{12}
\tightlist
\item
  Using the \texttt{unique} function, determine how many plots were
  sampled at Niwot Ridge. How is the information obtained from
  \texttt{unique} different from that obtained from \texttt{summary}?
\end{enumerate}

\begin{Shaded}
\begin{Highlighting}[]
\CommentTok{#Comparing the `unique` and `summary` functions}
\KeywordTok{unique}\NormalTok{(Litter}\OperatorTok{$}\NormalTok{plotID)}
\end{Highlighting}
\end{Shaded}

\begin{verbatim}
##  [1] NIWO_061 NIWO_064 NIWO_067 NIWO_040 NIWO_041 NIWO_063 NIWO_047
##  [8] NIWO_051 NIWO_058 NIWO_046 NIWO_062 NIWO_057
## 12 Levels: NIWO_040 NIWO_041 NIWO_046 NIWO_047 NIWO_051 ... NIWO_067
\end{verbatim}

\begin{Shaded}
\begin{Highlighting}[]
\KeywordTok{summary}\NormalTok{(Litter}\OperatorTok{$}\NormalTok{plotID)}
\end{Highlighting}
\end{Shaded}

\begin{verbatim}
## NIWO_040 NIWO_041 NIWO_046 NIWO_047 NIWO_051 NIWO_057 NIWO_058 NIWO_061 
##       20       19       18       15       14        8       16       17 
## NIWO_062 NIWO_063 NIWO_064 NIWO_067 
##       14       14       16       17
\end{verbatim}

\begin{quote}
Answer: There were 12 plots sampled at Niwot Ridge. When used on one
column of a data frame, \texttt{unique} eliminates duplicate values in
the column and returns one of each unique value, while \texttt{summary}
returns each unique value along with its frequency in the column of
data.
\end{quote}

\begin{enumerate}
\def\labelenumi{\arabic{enumi}.}
\setcounter{enumi}{13}
\tightlist
\item
  Create a bar graph of functionalGroup counts. This shows you what type
  of litter is collected at the Niwot Ridge sites. Notice that litter
  types are fairly equally distributed across the Niwot Ridge sites.
\end{enumerate}

\begin{Shaded}
\begin{Highlighting}[]
\KeywordTok{ggplot}\NormalTok{(Litter, }\KeywordTok{aes}\NormalTok{(}\DataTypeTok{x =}\NormalTok{ functionalGroup)) }\OperatorTok{+}\StringTok{ }
\StringTok{  }\KeywordTok{geom_bar}\NormalTok{() }\OperatorTok{+}
\StringTok{  }\KeywordTok{labs}\NormalTok{(}\DataTypeTok{x =} \StringTok{"Functional Group"}\NormalTok{)}
\end{Highlighting}
\end{Shaded}

\includegraphics{A03_DataExploration_files/figure-latex/unnamed-chunk-11-1.pdf}

\begin{enumerate}
\def\labelenumi{\arabic{enumi}.}
\setcounter{enumi}{14}
\tightlist
\item
  Using \texttt{geom\_boxplot} and \texttt{geom\_violin}, create a
  boxplot and a violin plot of dryMass by functionalGroup.
\end{enumerate}

\begin{Shaded}
\begin{Highlighting}[]
\CommentTok{#Boxplot}
\KeywordTok{ggplot}\NormalTok{(Litter) }\OperatorTok{+}
\StringTok{  }\KeywordTok{geom_boxplot}\NormalTok{(}\KeywordTok{aes}\NormalTok{(}\DataTypeTok{x =}\NormalTok{ functionalGroup, }\DataTypeTok{y =}\NormalTok{ dryMass)) }\OperatorTok{+}
\StringTok{  }\KeywordTok{labs}\NormalTok{(}\DataTypeTok{x =} \StringTok{"Functional Group"}\NormalTok{, }\DataTypeTok{y =} \StringTok{"Dry Mass"}\NormalTok{)}
\end{Highlighting}
\end{Shaded}

\includegraphics{A03_DataExploration_files/figure-latex/unnamed-chunk-12-1.pdf}

\begin{Shaded}
\begin{Highlighting}[]
\CommentTok{#Violin plot}
\KeywordTok{ggplot}\NormalTok{(Litter) }\OperatorTok{+}\StringTok{ }
\StringTok{  }\KeywordTok{geom_violin}\NormalTok{(}\KeywordTok{aes}\NormalTok{(}\DataTypeTok{x =}\NormalTok{ functionalGroup, }\DataTypeTok{y =}\NormalTok{ dryMass), }\DataTypeTok{draw_quantiles =} \KeywordTok{c}\NormalTok{(}\FloatTok{0.25}\NormalTok{, }\FloatTok{0.5}\NormalTok{, }\FloatTok{0.75}\NormalTok{), }\DataTypeTok{scale =} \StringTok{"count"}\NormalTok{) }\OperatorTok{+}\StringTok{ }
\StringTok{  }\KeywordTok{labs}\NormalTok{(}\DataTypeTok{x =} \StringTok{"Functional Group"}\NormalTok{, }\DataTypeTok{y =} \StringTok{"Dry Mass"}\NormalTok{)}
\end{Highlighting}
\end{Shaded}

\begin{verbatim}
## Warning in regularize.values(x, y, ties, missing(ties)): collapsing to
## unique 'x' values

## Warning in regularize.values(x, y, ties, missing(ties)): collapsing to
## unique 'x' values

## Warning in regularize.values(x, y, ties, missing(ties)): collapsing to
## unique 'x' values
\end{verbatim}

\includegraphics{A03_DataExploration_files/figure-latex/unnamed-chunk-12-2.pdf}

Why is the boxplot a more effective visualization option than the violin
plot in this case?

\begin{quote}
Answer: Because we are showing the dry mass within each of eight
different functional groups, each individual violin plot does not show
large enough probability frequencies to make the plots appear as
anything much more than a line. The data would need to be divided into
fewer groups for the probability frequency aspect of the violin plot to
be informative.
\end{quote}

What type(s) of litter tend to have the highest biomass at these sites?

\begin{quote}
Answer: Needles have the highest biomass, followed by mixed types of
litter and twigs/branches (there are some individual samples of woody
material that have relatively high dry mass, although woody material as
a whole does not).
\end{quote}


\end{document}
